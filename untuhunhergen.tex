\documentclass{article}
\usepackage[UTF8]{ctex}
\usepackage{tabularx}
\usepackage{hyperref}
\usepackage{lmodern}
\usepackage{multicol}
\usepackage{fullpage}
\usepackage{amsmath}
\usepackage[bottom]{footmisc}


\title{重刻清文虛字指南編\\dasame foloho manju gisun -i untuhun hergen -i temgetu jorin bithe}
\author{張卓暉 錄入}
\begin{document}
    \maketitle

\tableofcontents

\section{manju gisun -i untuhun hergen -i temgetu jorin bithe \v{s}utucin\\
重刻清文虛字指南編 序}

\subsection{\v{S}utucin\\序}
manju bithe ocibe, nikan bithe ocibe, gemu doro be ejere tetun, julgei gisun okini, te -i gisun okini, yooni bithe be ulara baitalan. kungz'i -i henduhengge, gisun de \v{s}u ak\={u} oci, ularangge goidarak\={u} sehe be tuwaci, doro bithei iletun de genggiyelere be dahame, bithe inu gisun -i iletun de iletulembikai. te bici, ihan be cukubuhe, tura de jalukiyahangge, bithe cagan ton ak\={u}, tuttu seme, nikan bithe udu ambula bicibe, ulhime bahanahai \v{s}uwe hafuka de, bihe bihei ini cisui ereci tede hafunara ferguwecuke babi. manju bithe de oci tuttu ak\={u}. emu gisun -i sark\={u} bithe bici, uthai emu gisun de icihe ombi. emu hergen -i ulhirak\={u} gisun bici uthai emu hergen de fijartun ombi. tere anggala manju bithei mangga ba, untuhun gisun de bisire dabala, yargiyan hergen de ak\={u}. deribure acabure kemun ici de, untuhun gisun ak\={u} oci, terei jurgan be iletuleme muterak\={u}. holboro ulire sudala siren de, untuhun gisun ak\={u} oci, terei g\={u}nin be tuyembume banjinarak\={u}. tacire urse, terei kemun ici be baharak\={u} oci, deribure acabure babe ilgame muterak\={u}. sudala siren be sark\={u} oci, holboro ulire babe getukeleme muterak\={u}. tere ubalyambuha sunja nomun duin bithe -i jergi geren bithe de, untuhun gisun yargiyan jurgan be gemu yongkiyabuha bicibe, damu debtelin jaci labdu, emu erinde biretei ak\={u}name tuwame banjinarak\={u}. tede tuktan tacire urse de tusa ojoro hergen tomorhon gisun getuken -i bithe be baiki seci, yala absi mangga inu. ubabe g\={u}ninaha daci, hercun ak\={u} ede g\={u}nin a\v{s}\v{s}afi, tereci mini mentuhun albatu be bodorak\={u}, beyei heni saha babe tucibume, manju gisun -i untuhun heregen -i ucun be majige tuwancihiyame dasatafi, gisun tome hergen aname, duibulen gisun be yarume dosimbufi, emu bithe obume banjibufi, temgetu jorin seme gebulehe, ubaliyambure be tuktan tacire urse, erebe gaifi fuha\v{s}ame sibkime ohode, heo seme alin be tafara doko yenju, mederi be doore dogon tuhan obuci ombi dere. damu mini tacihangge cinggiya bahanahangge tongga. ere bithe be banjibure de, erebe dosimbu nak\={u} terebe melebure ufaracun ak\={u} ume muterak\={u} ayoo sembi. damu buyerengge den genggiyen saisa ere bithe be sabufi ambula tuwancihiyame, tesuhek\={u} babe niyecetere, hamirak\={u} babe dasatara ohode, emu gisun nonggibuci, emu gisun -i tusa bahara. emu hergen eberembuci, emu hergen -i jelen geterere be dahame, ere mini jab\v{s}an sere angggala, mini amaga tacire ursei jab\v{s}an kai.\\

\noindent badarangga doro -i juwanci aniya niowanggiyan bonio ulgiyan biya ice de, \emph{heo tiyan wanfu} gingguleme ejeme araha.\\

文無清漢,皆為載道之輿;辭有古今,悉屬傳文之具。孔子曰,言之無文,行之不遠。是道以文明,而文亦由辭而顯也。夫汗牛充棟,載籍靡窮,然漢文雖博,而融會貫通,久之自有即此達彼之妙。清文則弗然,有一句不知,即為一句之疵,句有一字未解,即為一字之玷。且清文之難,純在虛文,而不在實字。起合之準繩,非虛文無以達其義,貫串之脈絡,非虛文弗克傳其神。學者不得其準繩,則起合無以辨,不知其脈絡,則貫串無由明。其五經四子繙譯諸書,虛文實義,無所不備,然卷帙繁多,一時徧覽難周。求其字簡,句易有裨於初學者。戛戛乎難之,每念及此,不覺有感於斯。於是不揣鄙陋,樗櫟自專,取清文虛字歌,稍加潤色,逐句逐字,引以譬語,集成一帙,命之曰指南。初學繙譯者誠取而熟玩之,可為梯山之捷徑,航海之津梁也與。第以余之葑菲芻蕘,集是編也,不無引此失彼之弊。惟願高明遇此,大為筆削,補其不足,匡其不逮。則多一言,有一言之益,去一字,少一字之憾,豈余之幸也,亦來許之幸也。\\

\noindent 光緒十年甲申陽月朔\\
\noindent 厚田萬福謹識

\subsection{\v{S}utucin\\序}
gung \v{s}u z'i umesi faksi seme, \emph{erguwejitu durbejitu} (規矩) ci uksalame muterak\={u}. kumun -i hafan kuwang dembei galbi seme, alioi elioi be kemun obuhabi. \v{s}u muten de badarambuci, ya hacin uttu waka ni. tuttu ofi \v{s}u fiyelen arambi sere doro, jilgan mudan durun kemun tuktan tacin urse urunak\={u} narh\={u}\v{s}ame sibkici acambi. sain eldengge gincihyan yangsangga, inenggi goidatala kiceci, ini cisui isiname mutembi. damu kimcici, nikan bithei duka de dosire kooli. tuktan tacire urse de tutabuha eiten fiyelen. ududu minggan tangg\={u} hacin -i teile ak\={u}. musei manju gisun -i mujilen -i turu, fuhali ulanduha bithe komso ofi, utala inenggi taciha seme, hono terei jorin be getukelehek\={u} de isinahabi. bi kemuni manju gisun -i toktooho kooli be isamjafi, bithe arame amaga tacire urse de tusa araki seme g\={u}nihade, beye ulhihengge cinggiya sahangge komso babe tulbime, gelhun ak\={u} cihai salihak\={u}, tere inenggi emu yamun -i gucu \emph{wan agu} -i emgi, ere turgun be leoleme isiname ofi, \emph{siyan \v{s}eng} emu bithe be tucibufi minde tuwabuha de, ere manju gisun -i temgetu jorin bithe, mini gucu de folobuha sehebi. bi debtelin neifi h\={u}lara de, buyehei sindame jenderak\={u}, \emph{siyan \v{s}eng} ni g\={u}nin, ainci neneme mini mujilen be bahabi. damu emu niyalma ci banjibufi. largin fulu kemungge dulem\v{s}eku -i ufaracun be ak\={u} obume muterak\={u}. utala niyalma de sarkiyabufi, talude calabume ta\v{s}arabuha calabun bi. tereci g\={u}nin de getuken ak\={u} ba bici, tere tokome faksalabuki, giyan de narh\={u}n ak\={u} ba bici, tere tokome subuki geli tesurek\={u} de niyecehe, dalhidara babe meitehe, tuttu seme urunak\={u} \emph{siyan \v{s}eng} inu sehe manggi, teni bithe de arahabi. sahah\={u}n meihe -i niyengniyeri forgon de, \emph{gioi jeng tang} sere bithe uncara puseli -i niyalma, dasame foloro mujin bi. mini asaraha debtelin be baifi, bi musei manju bithe be badarambure be sai\v{s}ambime, geli musei amaga tacire urse de tusa arara be maktaha turgunde, tuttu kimcime acabufi, folobume \v{s}uwaselabuha, gebu be nonggime \emph{dasame foloho manju gisun -i untuhun hergen -i temgetu jorin bithe} seme gebulehe. emu aniya otolo bithe \v{s}anggaha de, bi tereci \v{s}utucin arafi da dube be ejehe.\\

\noindent badarangga doro -i orin aniya niyengniyeri ujui biyade ujen cooha g\={u}sai \emph{fung\v{s}an} \v{s}utucin araha.\\

夫公輸至巧,規矩難離,師曠極聰。律呂是式,推之於文藝,何莫非然。是以行文一途,格調機局,初學務宜切講。精英藻麗,工久自能造詣也。惟考漢文入門之法,示初學諸篇,無慮數千百種。而吾三韓心法,竟鮮傳書。以致為學多日,尚有未能明其旨者。余嘗思將清文成法,集而為書,以益後進。而自問淺識寡見,未敢字專。一日與同寅萬公,語及其故,先生出一卷示余曰,此清文指南,友人某已梓之矣。余展卷讀之,愛不釋手。先生之意,蓋先有獲於我心者也。惟是編諸一手,不無繁衍簡略之虞。鈔假多人,間有魯魚亥豕之弊。於是意有未明,靣請剖之,理有未詳,靣請解之。又且補其不足,去其重複,然必先生韙之然後筆之於書也。癸巳春,聚珍坊主,有志重梓,索余藏本。余既嘉其振我國書,復羨其益我後學。故加校核,以付剞劂,增其名曰,重刻清文虛字指南編。歷寒暑而書成,余遂序之以誌其原委。\\

\noindent 光緒二十年孟春之月漢軍鳳山序

% \subsection{Uheri Ton 卷上}

%      1. de; be bu, de bu; -i, ni.
%      5. ci, deri; sa, se, ta, te, sei; ji, ju, nu, ca; ai, ao, ak\={u}n; kai; dabala.
%      10. ume; hono bade; rak\={u}, ayoo; seme, de seme, be seme; udu; ocibe; anggala.
%      15. aibi; canggi, teile, adali, gese; esi; jakade; bihe; ombi, oci, ome.
%      20. fi, me, me fi, pi; saka, jaka; me uthai; mbihede; fi dere, fi kai.
%      25. ohode, fibi, mebi; hebi, habi; kek\={u}, rak\={u}; hai, hei; ofi; waka; rangge, rengge; ci banjihangge.
%      30. ofi kai; de kai; be kai; henduhengge; sehengge, sembi; ainci, dere;
%      35. bime (凡屬bime者皆入此); gojime; ralame, releme; aika, yala, aikabade; talude; ainambi; acambi, acanambi; mutembi, bahanambi.
%      40. antaka; nam\v{s}an; dade, daci,elekei; manggi; andande, sidende, nerginde; tetendere, be dahame; onggolo, unde; nggala, nggele; uttu, ohode;
%      45. adarame.

% 下卷目錄終.

% \subsection{Uheri Ton 卷下}

%      1. maka, dele; aise; manggai; seci, sehe; obu, oso; ki, bai, dere.
%      5. ki sembi; rao, reo; de ombi; haminambi, hamimbi; amuran, guwelke; sark\={u}, baiburak\={u}, joo; kini; ai hacin -i; eitereme; cini; donjici; absi;
%      10. cohome; bi, ak\={u}; de bi, de ak\={u}; mangga, ja; yaya; inu; giyan; kecine.
%      15. ele, hale, hele; ai dalji; nurh\={u}me; nisihai; noho; muterei teile; ebsihe; manggi, nak\={u}; mari, nggeri, mudan; dari, tome; aname; kan, ken, kon;
%      20. meliyan; -sh\={u}n, -shun; -tai, -tei; -ngga, -ngge, -nggo; gelhun ak\={u}; alimbaharak\={u}; hercun ak\={u}; g\={u}nihak\={u}; eberi ak\={u}; banjinarak\={u}; umainaci ojorak\={u}; esi seci ojorak\={u}; urunak\={u}; urui.
%      25. giyanak\={u}; ambula; cohotoi; daljak\={u}; heni, majige; an -i, da an -i; jing; jingkini; doigonde; ishunde; ildun de; ini cisui; elemangga, neneme; uthai; eiterecibe;
%      30. emke emken -i; \v{s}uwe; cinggai; mini teisu; -tala, -tele, -tolo; ome, ofi, ojoro; oho, ombi, obure; oci, ome, ofi; uttu ofi; tuttu ofi; tere anggala; tuttu seme;
%      35. te bici, bicibe; eici embici; jalin, haran, turgun; ainu; ereni, ede, tereci; naranggi, jiduji; umesi (凡umesi之屬皆入此); asuru; emdubei.
%      40. umai, fuhali; gemu, yooni, wacihiyame; ci tulgiyen; ala, jabu, fonji; dele, wala (凡-le, -la之屬皆入此).
%      45. emgi, sasa, baru; cihai, funde; ici, songkoi; -hede, -rede; -mbi; -ra, -re; baha; eicibe; adarame, ainahai; mujangga; jai, geli, sirame;
%      50. kemuni, teni, damu, wajiha, wajihabi.

% 下卷目錄終.

\section{dasame foloho manju gisun -i untuhun hergen -i temgetu jorin bithe\\重刻清文虛字指南編 上\\\large{蒙古\;萬福厚田\;著\\
漢軍\;鳳山禹門\;訂}}
\noindent -i ni ci kai與de be,用處最廣講論多。\\
\noindent manju bithede untuhun hergen baitalara ba umesi labdu, tacire urse g\={u}nin were\v{s}eme sibkime baici acambi.\\
\noindent 清文用虛字處最多,學者當留心講求之。\\

\subsection{與格格助詞de}
\noindent 裏頭上頭併時候,給與在於皆是de。
\begin{center}
    \begin{tabularx}{\textwidth}{XX}
        angga de h\={u}lambi, mujilen de ejembi. &
        口裏頭念,心裏頭記。\\
         uju de huk\v{s}embi, falangg\={u} de tukiyembi. & 頭上頭頂之,掌上頭擎之。\\
         hafan tere de bolgo oso, baita icihiyara de tondo oso.& 做官的時候要清,辦事的時候要公。\\
         minde gajifi tuwa, inde benefi \v{s}a.&給我拿來看,給他送去瞧。\\
         niyalma de tusa arambi, beye de jab\v{s}aki bahambi.&與人方便,與己得益。\\
         aibade acambi, ubade aliyambi.&在何方相見,在此處等候。\\
         baita de kicembi, gisun de olho\v{s}ombi.&敏於事,慎於言。
    \end{tabularx}
\end{center}

\subsection{及物動詞的賓格be, 使動中綴-bu-動詞接賓格...be...V-bu-}
\noindent 把、將、以、使、令、教字,共是七樣盡繙be。下邊必有-bu-字應,不然口氣亦可托。
\begin{center}
    \begin{tabularx}{\textwidth}{XX}
        jingse be hadabumbi, g\={u}lha be etumbi.&把頂子帶之,把鞾(靴)子穿之。\\
        bithe be ureme h\={u}la, hergen be saikan ara.&將書熟熟的念,將字好好的寫。\\
        gosin jurgan be fulehe obumbi, doro erdemu be tuhe obmbi.&以仁義為本,以道德為歸。\\
        ambasa saisa be an be tuwakiyabumbi, buya niyalma be waka be ulhibumbi.&使君子安常,使小人知非。\\
        niyalma be h\={u}nin usambumbi, jaka be teisu bahabumbi.&令人失望,令物得所。\\
        amban oho niyalma be tondo okini sembi, jui oho niyalma be hiyoo\v{s}un okini sembi.&教臣子忠,教人子孝。\\
        niyalma be sain gisun gisurebumbi, niyalma be sain baita yabubumbi.&教人說好話,教人行好事。
    \end{tabularx}
\end{center}

\subsection{使動動詞V-bu-前無賓格格助詞的情形}
\noindent -bu-字亦有自然處,祇看有be與無be。
\begin{center}
    \begin{tabularx}{\textwidth}{XX}
     erdemu ilibuha manggi, gebu mutebumbi; beye tuwacihiyabuha manggi, boo teksilebumbi.&德建而後名立,身修而後家齊。
    \end{tabularx}
\end{center}

\subsection{表示話題、解釋的...serengge...be構造}
\noindent 也字繙be講是字,上用serengge與sehengge。
\begin{center}
    \begin{tabularx}{\textwidth}{XX}
     tondo serengge, dulimba be; ginggun serengge, cibsen be.&忠者,中也;敬者,靜也。\\
     banin serengge, uthai giyan be; giyan serengge, uthai sukdun be.&性即理也,理即氣也。\\
     fulehe be kicembi sehengge, hiyoo\v{s}un deocin be yabure be; \v{s}u be tacimbi sehengge, irgebun bithe be kicere be.&所謂務本者,是行孝弟也;所謂學文者,是攻詩書也。
    \end{tabularx}
\end{center}

\subsection{被動動詞...de...V-bu-}
\noindent 上de下-bu-為被字,用法如同上有be。
\begin{center}
    \begin{tabularx}{\textwidth}{XX}
     weri de basubumbi, g\={u}wa de gidabumbi.&被人恥笑,被人欺壓。
    \end{tabularx}
\end{center}

\subsection{屬格格助詞-i, ni}
\noindent 的之以用皆繙-i,在人運用要斟酌。
\begin{center}
    \begin{tabularx}{\textwidth}{XX}
     fe ice -i baita hacin, uru waka -i arbun dursun.&新的案件,是非的情形。\\
     duin mederi -i onco, tumen jaka -i largin.&四海之廣,萬物之多。\\
     tondo -i ejen be uilembi, onco -i fejergingge be kadalambi.&以忠事君,以寬御下。\\
     gida -i gidalambi, \v{s}aka -i \v{s}akalambi.&用槍扎,用叉扠。
    \end{tabularx}
\end{center}

\noindent 四頭之下-i念ni,此外單連都使得。
\begin{center}
    \begin{tabularx}{\textwidth}{XX}
     gocish\={u}n anah\={u}njan -i tacin, tondo akdun -i yabun.&遜讓之風,忠信之行。\\
     giyan fiyan -i gamambi, funcen daban -i icihiyambi.&調停齊楚,辦理裕如。\\
     abkai \v{s}u na -i giyan, niyalma buyenin jakai arbun.&天文地理,人情物態。\\
     yasai tuwambi, galai jorimbi.&用眼瞧,用手指。
    \end{tabularx}
\end{center}

\noindent 五頭的之字繙ni,猶同-i字一樣說。
\begin{center}
    \begin{tabularx}{\textwidth}{XX}
     ya wang ni harangga, ya gung ni duka.&那王的屬下,那公的門上。\\
     \emph{ba wang} ni baturu, \emph{jang liyang} ni mergen.&霸王之勇,張良之智。
    \end{tabularx}
\end{center}

\noindent 句尾之ni呢、哉用,上承文氣有所托。

\subsection{疑問詞adarame, ainahai; ainu, ai, aiseme}
\begin{center}
    \begin{tabularx}{\textwidth}{XX}
     ere ai jaka ni? bi ainu takarak\={u} ni?&這是什麼東西呢?我為何不認得呢?\\
     ere baita ainahai foihori biheni? erei giyan ainahai ta\v{s}an ni?&其事豈偶然哉?其理豈或爽哉?\\
     adarame oncodome guwebume mutembini? ainahai weilen de fangkabuci ombini?&何能寬宥,豈足蔽辜?\\
     ainu uttu manda ni? aiseme tuttu ek\v{s}embini?&因何如此慢,怎麼那麼忙?
    \end{tabularx}
\end{center}

\subsection{離格、比格格助詞ci}
\noindent ci字若在整字下,自從由、第、比、離說。否則上連破字用,``若''字、``若是''還有``則''。
\begin{center}
    \begin{tabularx}{\textwidth}{XX}
     abka ci wasimbi, na ci banjimbi.&自天而降,自地而生。\\
     julgeci tetele gemu uttu, ereci amasi geli antaka.&從古至今皆如是,從此以往復如何。\\
     hanciki ci goroki de isinambi, dorgi ci tulergi de hafunambi.&由近及遠,由中達外。\\
     uduci & 第幾\\ uducingge& 第幾的\\ilaci&第三\\ilacingge&第三的\\
     ubaci goro ak\={u}, tubaci umesi hanci.&離這裡不遠,離那裡很近。\\
     i sinci fulu, bi weci eberi? &他比你強,我比誰不及。\\
     geneci uthai h\={u}dun jio, jiderak\={u} oci aliyara be joo.&若去就快來,若不來不必等。\\
     teike teisulehengge weci? tenike tunggiyehengge aici?&將纔遇見的是誰人,方纔拾著的是什麼。\\
     kiceme tacici \v{s}anggaci ombi, tacirak\={u} oci beyebe waliyabumbikai. &勤學則可成,不學則自棄也。
    \end{tabularx}
\end{center}

\subsection{離格格助詞deri}
\noindent deri亦是從由字,比ci實在有著落。
\begin{center}
    \begin{tabularx}{\textwidth}{XX}
     ederi absi genembi? tederi boode marimbi.&從此處何往?從那裏回家。\\
     muke \v{s}eri deri tucimbi, edun sangga deri dosimbi.&水由泉出,風由孔入。\\
     fa deri gala be jafambi, mederi deri bira de juwembi.&自牖執其手,自海運於河。
    \end{tabularx}
\end{center}

\subsection{複數後綴sa, se, si, ta, te,複數的屬格和賓格}
\noindent sa, se, si, ta, te們衆等,祇看本字口氣說。官事又有二樣用,sai字與be俱使得。等事等物與等處,乃用jergi方使得。
\begin{center}
    \begin{tabularx}{\textwidth}{XX}
     han sa &君等\\ wang sa &王等\\gung se&公等\\ambasa&大臣等\\hafasa&官員等\\irgese&民人等\\mergese&賢士們\\hahasi&衆男子\\hehesi&衆女人\\amjita&伯父們\\ah\={u}ta&衆兄們\\eshete&叔父們\\deote&衆弟們\\
     aliha bithei da ortai sei gingguleme wesimbuhe.&大學士鄂爾泰等謹題。\\
     amban(ambasa?) be baicaci, jab\v{s}aha ufaraha jergi baita.&臣等查,得失等事。\\
     eture jetere jergi jaka, hanci goro -i jergi ba.&衣食等物,遠近等處。
    \end{tabularx}
\end{center}

\subsection{“來”、“去”中綴V-ji-, V-ju-, V-na-, V-ne-,及其特殊命令式}
\noindent -ji, -ju是來, -na, -ne是去。
\begin{center}
    \begin{tabularx}{\textwidth}{XX}
     dosinjirengge dosinju, tucinjirengge tucinju.&進來的進來,出來的出來。\\
     dosinarangge dosinu, tucinerengge tucinu.&進去的進去,出去的出去。\\
     dosifi fonjina, tucifi gisurene.&進去問去,出去說去。
    \end{tabularx}
\end{center}

\subsection{互動中綴V-nu-, V-du-,齊動中綴V-ca-, V-ce-}
\noindent 齊衆-nu-、-du-與-ca-、-ce-。
\begin{center}
    \begin{tabularx}{\textwidth}{XX}
     sasa acanumbi, uhei hebdenumbi.&大家齊見,公同齊商。\\
     ishunde aisilandumbi, geren tem\v{s}endumbi.&一齊相助,大家齊爭。\\
     dalbade ilicambi, amala dahacambi.&齊在旁邊站,一齊後面跟。\\
     ekisakai tecembi, sula leolecembi.&齊都靜坐,大衆閒談。
    \end{tabularx}
\end{center}

\subsection{加強/頻率中綴-t/dA-,-\v{s}A-, -cA-, -jA-}
\noindent -ta-, -da-, -te-, -de-, -\v{s}a-, -\v{s}e-, -\v{s}o-, -ca-, -ce-, -co-與-ja-, -je-, -jo-,此等字樣皆一意,頻頻常常連連說。
\begin{center}
    \begin{tabularx}{\textwidth}{XX}
     ai\v{s}ilatambi &常常幫助\\uba\v{s}atambi&頻頻反覆\\ucudambi &連連攪合\\baktadambi &常常容納\\niyecentembi&頻頻補綴\\h\={u}ngsitambi&連連摔抖\\ibedembi&連連進步\\seoledembi &常常思索\\habta\v{s}ambi&連連展眼\\bilu\v{s}ambi &頻頻摩拊\\elde\v{s}embi&頻頻照耀\\neme\v{s}embi&連連增加\\toko\v{s}ombi &連連戮刺\\holto\v{s}ombi&常常撒謊\\arganacambi&頻頻生芽\\ataracambi&常常喧鬧\\\v{s}urgecembi&常常打顫\\ergecembi&常常安逸\\korsocombi&頻頻愧恨\\hoksoncombi&常常煩愠\\g\={u}ninjambi&常常思想\\toohanjambi&常常猶豫\\erehunjembi& 連連指望\\ulhinjembi&常常醒悟\\golonjombi&頻頻恐怕\\olhonjombi&常常畏懼。
    \end{tabularx}
\end{center}

\subsection{疑問詞ai}
\noindent ai字多貼何字講,又是什麼與什麼。
\begin{center}
    \begin{tabularx}{\textwidth}{XX}
     ai baita& 何事\\ai turgun&何故\\ai dalji&何干\\ai sui&何苦\\ai g\={u}nin&什麼心意\\ai yabun 什麼行為,ai bengsen 什麼本事,ai muten 什麼能柰,ai yokto 什麼趣兒,ai demun 什麼樣子。
    \end{tabularx}
\end{center}

\subsection{疑問語氣詞-ao, -eo, -io}
\noindent -ao, -eo, -io尾皆疑問,又作反口語氣說。
\begin{center}
    \begin{tabularx}{\textwidth}{XX}
     tere niyalma be takambio? &認得那個人麼?\\
     ere giyan be ulhimbio?& 懂得這個理麼?\\
     julgei ulasi be sabuhao? &瞧見古蹟了麼?\\
     yoro gisun be donjihao? &聽見謠言了麼?\\
     cananggi si geneheo? &前日你去了麼?\\
     sikse i jiheo? &昨天他來了麼?\\
     aika amtanggao? &有什麼有趣兒嗎?\\
     yala sebjenggeo? &真是個樂兒麼?\\
     dule uttu nio? &原來這樣麼?\\
     mini teile nio? &豈只是我嗎?\\
     ereci tucinerengge bio? &有外此者乎?\\
     ereci dulenderengge bio? &豈加於此哉?\\
     gosin seci ombio? &可謂仁乎?\\
     targacun oburak\={u}ci ombio? &可不戒歟?\\
     inu sebjen wakao? &不亦樂乎?\\
     iletu tusa wakao? &非明效歟?\\
     aika ede ak\={u} semeo? &顧不在茲乎?\\
     yala sain ak\={u} semeo? &豈不美哉?
    \end{tabularx}
\end{center}

\subsection{否定疑問語氣詞-rak\={u}n, -hak\={u}n, ak\={u}n, -y\={u}n, -\v{s}on, -giy\={u}n}
\noindent 反詰不曾未曾語,-rak\={u}n、-hak\={u}n等字多;倘若本文是整字,則用ak\={u}n下邊托。
\begin{center}
    \begin{tabularx}{\textwidth}{XX}
     baitalaci ojorak\={u}n? &不可用麼?\\
     yabume muterak\={u}n? &不能行麼?\\
     yala sabuhak\={u}n? &真沒瞧見麼?\\
     umai donjihak\={u}n?& 並未聽見麼?\\
     muse gisurehek\={u}n? &偺們沒言說麼?\\
     suwe heb\v{s}ehek\={u}n? &你們未商量麼?\\
     tacihangge sain ak\={u}n? &學的不好麽?\\
     urebuhengge mangga ak\={u}n? &練得不強麼?
    \end{tabularx}
\end{center}

\noindent 四頭整字變疑問,-y\={u}n、-\v{s}on、-giy\={u}n字尾接著。
\begin{center}
    \begin{tabularx}{\textwidth}{XX}
     ere ucuri saiy\={u}n? &這一向好麽?\\
     ini gisun ta\v{s}on? &他的話假麼?\\
     tere baita yargiy\={u}n? &那宗事真麼?
    \end{tabularx}
\end{center}

\subsection{決斷語氣詞kai, “罷、耳”V-rA/hA dabala}
\noindent kai字落腳即決斷,繙作哪、呀、啊也說。
\begin{center}
    \begin{tabularx}{\textwidth}{XX}
     tuwaci mujangga kai. &看起來果然哪。\\
     gisurehengge yala ta\v{s}an ak\={u} kai. &所言真不假呀。\\
     toktoho doro kai. &一定的道理啊。\\
     ini cisui banjinara giyan kai. &自然之理也。
    \end{tabularx}
\end{center}

\noindent dabala本是``罷''字意,又作``耳''字也使得。或然又作而已矣,-hA、-rA、-k\={u}尾接處多。
\begin{center}
    \begin{tabularx}{\textwidth}{XX}
     angga canggi gisurere dabala. &竟嘴說罷。\\
     manggai niyalma be tafulara dabala. &不過勸人罷。\\
     niyalma tome beye de wesihun ojorongge bi, g\={u}nirak\={u} dabala. &人人有貴於己者,弗思耳。\\
     niyalma ainahai eterak\={u} de jobumbini, yaburak\={u} dabala. &夫人豈以不勝為患哉?弗為耳。\\
     tere gosin inu urebure de bisire dabala. &夫仁亦在乎熟之而已矣。\\
     ambasa sain kooli be yabume hesebun be aliyara dabala. &君子行法以俟命而已矣。
    \end{tabularx}
\end{center}

\subsection{否定祈使、命令ume, ”尚且“hono bade}
\noindent ume是休、勿、莫、別、毋,下用-ra, -re, -ro字托。
\begin{center}
    \begin{tabularx}{\textwidth}{XX}
     ume banuh\={u}\v{s}ara &休懶惰\\
     ume sartabura &勿遲悞\\
     ume oihorilara &莫輕忽\\
     ume onggoro &別忘記\\ume ginggun ak\={u} ojoro &毋不敬
    \end{tabularx}
\end{center}

\noindent hono bade是尚且,下有何況be ai hendure。若無反詰何況處,babi與ni俱可托。還有下接-mbi處,臨文隨地細斟酌。
\begin{center}
    \begin{tabularx}{\textwidth}{XX}
     bithe be hono giyangname bahanarak\={u} bade, \v{s}u fiyelen arara be ai hendure?&書尚且不會講,何況作文章?\\
     jalan -i encu demun -i niyalma, kungz'i mengz'i -i bithe be hono ulhime muterak\={u} bade, looz'i juwang z'i -i tacin de urui sibkime fuha\v{s}aci, ere bolgo getuken -i jalan be waliyafi, butu buruhun -i jug\={u}n de dosinaha be dahame, ai tusa arara babi?&世之異端之士,於孔孟之書尚不能解,而於老莊之術一味鑽研,是舍清明之世而入幽暗之途,有何益哉?\\
     tere tacin de hiyoo\v{s}un deocin be hono giyangnarak\={u} bade, gosin jurgan be ele weihuken secibe, ai doro be jafafi, abka na -i sidende ilibumbini?&
    其俗也孝弟尚且不講,仁義更以為輕,然以何道立於天地之間哉?\\
     hono urunak\={u} kiceme h\={u}sutulefi, dah\={u}n dah\={u}n -i hing seme girk\={u}ha manggi, teni bahara bade ede tacin fonjin -i doro, kicerengge ele nemeci, dosinarangge inu ele \v{s}umin ojoro be seci ombi.&
    尚必待功力專勤,至再至三而始得之,可以知學問之道,功愈加則業亦愈進。
    \end{tabularx}
\end{center}

\subsection{泛指各種情況bade}
\noindent ombio, semeo, mujanggao 三字上亦可用bade。
\begin{center}
    \begin{tabularx}{\textwidth}{XX}
     beye de tusa sere bade, g\={u}nin sith\={u}rak\={u}ci ombio.&尚且於己有益,可不專心麼?\\
     jalan -i baita hono uttu ojoro bade, inu nasacuka ak\={u} semeo.&世事尚然如是,不亦悲乎?\\
     emgeri kiceci, mujilen ulhire g\={u}nin bahanara bade, tacin fonjin \v{s}umin dosinarak\={u} mujanggao?&一勤則心領神會,學問豈不深造哉?
    \end{tabularx}
\end{center}

\subsection{“恐怕”V-rah\={u} ayoo}
\noindent -rah\={u} ayoo是恐字,連用-rah\={u}單ayoo。文氣斷住托sembi,串文句下用seme。
\begin{center}
    \begin{tabularx}{\textwidth}{XX}
     tacin fonjin eberi ojorah\={u} sembi,gungge gebu muteburak\={u} ayoo sembi.&恐其學問不及,恐怕功名不成。\\
     waliyarah\={u} seme g\={u}nin were\v{s}embi. efujere ayoo seme mujilen bibumbi.& 恐怕遗失而留神,恐其毀壞而在意。
    \end{tabularx}
\end{center}

\subsection{助動詞seme}
\noindent 單用seme是等、因等語,承上啟下為過脉。
\begin{center}
    \begin{tabularx}{\textwidth}{XX}
     uttu tuttu seme, harangga geren baci bithe benjihebi. encu afaha arafi ilgame faksalame sai\v{s}ame huwekiyebuki seme baime wesimbuhebi.&如此如彼等因由各該處咨行前來,另片奏請分別獎勵等語。
    \end{tabularx}
\end{center}

\noindent 上文結句連下用,中間過筆用seme。
\begin{center}
    \begin{tabularx}{\textwidth}{XX}
     coohai caliyan de eden ak\={u} seme anggala, ba na de yargiyan -i ambula tusa bi seme tuttu gelhun ak\={u} toktoho kooli be memereci ojorak\={u}. &不但與兵餉無虧,於地方實有裨益,故不敢拘泥成格。
    \end{tabularx}
\end{center}

\subsubsection{“雖在”...de seme}
\noindent de seme是雖在講。
\begin{center}
    \begin{tabularx}{\textwidth}{XX}
     \emph{hahi cahi} de seme inu aljaci ojorak\={u}. tuhere afara de seme kemuni ash\={u}ci ojorak\={u}.&雖在造次,亦不可離。雖在顛沛,仍不可去。
    \end{tabularx}
\end{center}

\subsubsection{“連是”...be seme}
\noindent be seme作連是說。
\begin{center}
    \begin{tabularx}{\textwidth}{XX}
     imbe seme, inu fimeci ojorak\={u}, simbe seme, geli ainara?&連他也惹不得,是你又怎麼樣?
    \end{tabularx}
\end{center}

\subsection{“雖然”udu...bicibe, V-hA seme, V-cibe, secibe, “總然”、“即便”、“無論”ocibe}
\noindent 雖字神情有udu,句下應有bicibe。-ha -he -ho等共整字,往往常有用seme。揣摩文氣如何語,-cibe, secibe, ocibe。
\begin{center}
    \begin{tabularx}{\textwidth}{XX}
     udu dabali -i gese bicibe, inu bahabuci acarangge. &雖覺過優,亦所應得。\\
     udu kooli de acanarak\={u} bicibe, inu acara be tuwame akduleci ombi.&雖不符例,亦可酌保。\\
     udu songkolome alh\={u}daha seme, hono hamikarak\={u} ayoo sembi.&雖然則效,猶恐不逮。\\
     udu badarambume fisembuhe seme, hono \v{s}uwe hafukak\={u}.&雖則衍述,尚未通徹。\\
     udu enduringge niyalma seme, inu sark\={u} babi.&雖聖人,亦有不知。\\
     udu bithe h\={u}lacibe, doro giyan be getukelerak\={u}.&雖然讀書,不明道理。\\
     udu gungge bi secibe, beyei g\={u}nin be dekder\v{s}eci ojorak\={u}.&雖然有功,不可自逞其志。\\
     udu endebuku ak\={u} secibe, ser seme babe olho\v{s}orak\={u}ci ojorak\={u}. &雖然無過,不可不慎其微。\\
     udu mujilen joborak\={u} ocibe, inu seoleme g\={u}nici acambi.&雖是不勞心,也得思索。\\
     udu h\={u}sun baiburak\={u} ocibe, inu katunjame hacihiyaci acambi.&雖然不費力,亦須勉強。
    \end{tabularx}
\end{center}
\noindent ocibe是總然即便,又作無論或是說。
\begin{center}
    \begin{tabularx}{\textwidth}{XX}
     uthai ici acarak\={u} ocibe, inu nikedeme baitalaci acambi.&即使不合式,亦得遷就著使。\\
     manju ocibe nikan ocibe, gemu hafan irgen dabala.&無論滿漢,皆屬臣民。\\
     jalafun ocibe aldasi ocibe, ya jalgan ton waka.&或壽或夭,孰非命算。
    \end{tabularx}
\end{center}

\subsection{“與其”anggala...“不如”de isirak\={u}}
\subsubsection{“與其...不如”V-rA anggala...de isirak\={u}}
\noindent “與其”繙為anggala,上連必用-ra, -re, -ro。下有不如不若字,用de isirak\={u}托。
\begin{center}
    \begin{tabularx}{\textwidth}{XX}
     mekele h\={u}lara anggala, dolori ejere de isirak\={u}.&與其竟念,不如心裏記。\\
     untuhun gisurere anggala, songkolome yabure de isirak\={u}.&與其空說,不如照著行。\\
     bahanarak\={u} ojoro anggala, amcame tacire de isirak\={u}.&與其不會,莫若趕著學。
    \end{tabularx}
\end{center}

\subsubsection{“不但、不惟”...sere anggala}
\noindent sere anggala是“不惟、不但”,又作“豈惟、豈但說”。
\begin{center}
    \begin{tabularx}{\textwidth}{XX}
     abka de teherembi seme anggala, inu na de jergileci ombi.&不惟參天,亦且兩地。\\
     hafan wesimbi sere anggala, geli ulin madambi.&不但陞官,且又發財。\\
     beye dursulembi seme anggala, geli h\={u}sutuleme yabumbi.&豈惟身體,而且力行。\\
     beye be tuwancihiyambi sere anggala, inu niyalma be dasambi.&豈但修己,抑且治人。
    \end{tabularx}
\end{center}

\subsection{aibi}
\subsubsection{“豈肯、豈能”V-rA aibi}
\noindent 豈肯、豈能用aibi,上面須用-ra, -re, -ro。
\begin{center}
    \begin{tabularx}{\textwidth}{XX}
     banjibume arara aibi? &豈肯造作?\\
     balai yabure aibi? &豈肯為非?\\
     ja -i onggoro aibi? &豈肯遽忘?\\
     muteburak\={u} aibi? &豈能不成?
    \end{tabularx}
\end{center}

\subsubsection{“何有”V-ci aibi}
\noindent -ci連aibi是何有。
\begin{center}
    \begin{tabularx}{\textwidth}{XX}
     dorolon anah\={u}njan -i gurun be dasame muteci, aibi? & 能以禮讓爲國乎?何有?
    \end{tabularx}
\end{center}

\subsubsection{“何妨”...de aibi}
\noindent 何妨aibi上連de。
\begin{center}
    \begin{tabularx}{\textwidth}{XX}
     gisun bici getukeleme gisurere de aibi?&有話何妨說明?
    \end{tabularx}
\end{center}

\subsection{“唯獨、僅上”[-i, de, be, V-rA, V-hA] canggi, teile}
\noindent “專、純、竟、只”使canggi,”僅上、唯獨“用teile。二字上接-i, ni, de, be字,-ra,-re,-ha,-he等字多。
\begin{center}
    \begin{tabularx}{\textwidth}{XX}
     oilorgi yangse be canggi wesihulembi. &專尚浮華。\\
     abkai giyan -i canggi. &純乎天理。\\
     sui araha canggi. &竟作孽。\\
     mangga arara canggi. &只逞強。\\
     ere emu niyalma teile. &僅此一人。\\
     ede teile wajimbio?. &止此而已乎?\\
     h\={u}dun be buyere teile. &惟有欲速。\\
     mini teile tuttu ak\={u}. &予獨不然。
    \end{tabularx}
\end{center}

\subsubsection{teile ak\={u}}
\noindent 若是不獨、不止句,teile ak\={u}語活多。
\begin{center}
    \begin{tabularx}{\textwidth}{XX}
     tacin fonjin fulu -i teile ak\={u}, yabun tuwakiyan inu sain. &不獨學問見長,品行也好。\\
     afame mutere teile ak\={u}, geli tuwakiyame bahanambi. &不止能戰,又且善守。
    \end{tabularx}
\end{center}

\subsection{“如同、相似”adali, gese}
\subsubsection{[-i, V-rA, V-hA] adali, gese}
\noindent adali如同、gese相似,總作比如似若說。二字之上接-i, ni,-ra,-re與-ha, -he諸字多。
\begin{center}
    \begin{tabularx}{\textwidth}{XX}
     ini arbun giru sini adali, sini banin wen ini gese. &他的形容如同你,你的品貌似乎他。\\
     jalafun julergi alin -i adali, h\={u}turi dergi mederi -i gese. &壽比南山,福如東海。\\
     sain be daharangge tafara adali, ehe be daharangge ulejere gese. &從善如登,從惡如崩。\\
     den alin de tafaha gese, \v{s}umin tunggu de enggelehe gese. &如登高山,如臨深淵。\\
     bimbime ak\={u} -i adali, jalu bime untuhun -i adali. &有若無,實若虛\\
     faijuma -i gese, tesurak\={u} -i gese. &似不妥,似不足。
    \end{tabularx}
\end{center}

\subsubsection{...de adali}
\noindent 與彼相同、與此相似,adali上又接de。
\begin{center}
    \begin{tabularx}{\textwidth}{XX}
     ini tacihangge sinde adali, sini arahangge inde adali. &他學的與你相同,你做的與他相似。
    \end{tabularx}
\end{center}


\subsection{“自然,正是”esi}
\noindent esi之下ci字托,短章粗語方使得。
\begin{center}
    \begin{tabularx}{\textwidth}{XX}
     esi jici&自然來\\esi geneci&自然去\\esi jeci&自然喫\\esi omici&自然喝
    \end{tabularx}
\end{center}

\subsection{引證已往的因果關係[V-rA, V-hA, -i] jakade}
\noindent jakade上用-ra,-re,ro,下文落腳-ha,-he,-ho。-ka,-ke,-ko皆一體用,引證已往述辭多。上邊若接-i, -ni字,又作跟前字句說。
\begin{center}
    \begin{tabularx}{\textwidth}{XX}
     i minde alara jakade, bi teni saha. &他告訴我,我纔知道了。\\
     si ejerak\={u} ojoro jakade, tuttu onggoho.& 你不記著,所以忘了。\\
     niyalma de hajilara jakade, tuttu algin algika. &親於親,所以揚名。\\
     dengjan dabure jakade, tuttu gehun gereke. &點上燈,所以大亮。\\
     bengsen fulu ojoro jakade, tuttu geren ci colgoroko. &本領高強,所以出眾。\\
     seibeni \emph{ioi} han amba muke be eberembure jakade, abkai fejergi necin oho. \emph{jeo gung} tulergi aiman be dahabufi, eshun gurgu be bo\v{s}oro jakade, tangg\={u} halai irgen nikton oho. kungzi \v{s}ajingga nomun be \v{s}anggabure jakade, facuh\={u}n amban h\={u}lha jui gelehe. &昔者禹抑洪水,而天下平。周公兼夷狄,驅猛獸,而百姓甯。孔子成春秋,而亂臣賊子懼。\\
     sefu -i jakade tacibure be baimbi, mini jakade yandume baimbi. &師傅的跟前討教,我等的跟前求情。
    \end{tabularx}
\end{center}

\subsection{已然bihe, “如果”bici}
\noindent 追論已然之未然,bihe bici下托bihe。
\subsubsection{對已然進行假設...bihe bici...bihe}
\begin{center}
    \begin{tabularx}{\textwidth}{XX}
     tacihak\={u} bihe bici, inu ulhime muterak\={u} bihe. &若未學過,也不能曉得。\\
     donjihak\={u} bihe bici, inu same muterak\={u} bihe. &若未聽見,亦不能知道。
    \end{tabularx}
\end{center}

\subsubsection{分詞接“如果”的用法V-hA bici...bihe}
\noindent -ha,-he等下接bici,下文末用bihe托。比論往事若如彼,也就如此意思說。
\begin{center}
    \begin{tabularx}{\textwidth}{XX}
     tere fonde yargiyan mujilen -i icihiyaha bici, inu calabuha de isinarak\={u} bihe. &當時若是認真辦,也不至舛錯。\\
     onggolo g\={u}nin were\v{s}ehe bici, te inu sark\={u} ume muterak\={u} bihe. &從前要留神,如今也不能不知道了。
    \end{tabularx}
\end{center}

\subsection{“可、不可”V-ci ombi/ojorak\={u}}
\noindent ombi ojorak\={u}為可、不可,上連ci字是法則。
\begin{center}
    \begin{tabularx}{\textwidth}{XX}
     ambasa saisa be unggici ombi, tuhebuci ojorak\={u}. holtoci ombi, geodebuci ojorak\={u}. &君子可逝也,不可陷也。可欺也,不可罔也。
    \end{tabularx}
\end{center}

\subsubsection{“如果...則”...oci...ombi}
\noindent 由此及彼言所以,oci下用ombi托。
\begin{center}
    \begin{tabularx}{\textwidth}{XX}
     dorolon jurgan be sark\={u} oci, beyebe ilibume muterak\={u} ombi. &不知禮儀,無以立身。\\
     tondo akdun be da obure oci, beye yabure de mangga ak\={u} ombi. &主乎忠信,不難行。
    \end{tabularx}
\end{center}

\subsubsection{"如果在"...de oci}
\noindent 若在、若往 de oci
\begin{center}
    \begin{tabularx}{\textwidth}{XX}
     dorgi de oci, gemun hecen be karmame dalimbi. &若在內,則拱衛京師。\\
     tulergi de oci, jase jecen be giyarime dasambi. &若在外,則撫巡邊疆。
    \end{tabularx}
\end{center}

\subsubsection{“如果將”...be oci}
\noindent be oci 若將、若把說。
\begin{center}
    \begin{tabularx}{\textwidth}{XX}
     ler ler seme bithe niyalma be oci, durun tuwak\={u} obume alh\={u}dara giyan. &若將藹藹丈人,宜其法為表範。\\
     hoo hoo seme coohai haha be oci, kalka hecen obume nikendure giyan. &若把赳赳武夫,應須依作干城。
    \end{tabularx}
\end{center}

\subsubsection{“成為”N ombi/ojoro}
\noindent o-字若要接整字,不作可字作為說。
\begin{center}
    \begin{tabularx}{\textwidth}{XX}
     ejen ofi ejen -i doro be ak\={u}mbuki, amban ofi amban -i doro ak\={u}mbuki seci, juwe de gemu \emph{yoo} han, \emph{\v{s}\={u}n} han be alh\={u}daci wajiha.& 欲為君盡君道,欲為臣盡臣道,二者皆法堯舜而已矣。
    \end{tabularx}
\end{center}

\subsubsection{名詞/形容詞接ome表示類似-lA-派生中綴的意思}
\noindent 整字難破接ome,猶同-la, -le用法活。
\begin{center}
    \begin{tabularx}{\textwidth}{XX}
     gucu gargan de akdun ome, ga\v{s}an falga de h\={u}waliyasun ome muteci, hanciki ci goroki de isitala, niyalma tome gemu gingguleme ujeleme jiramilame tuwambi kai. &能信乎朋友,和夫鄉黨,則由近及遠,人皆敬重而厚待之矣。
    \end{tabularx}
\end{center}

\subsection{分詞修飾名詞[-rA,-kA,-hA] N}
\noindent 整字上接-ra -re ro,-ka -ke -ko與-ha -he -ho。此等字作“的、之”用,又有“所”字意含著。
\begin{center}
    \begin{tabularx}{\textwidth}{XX}
        gurun be dasara fulehe. & 為國之本\\
        dasan be yabubure oyonggo. & 為政之要\\
        erdemu be sonjoro kooli. & 掄才之典\\
        gebu algika tacin. & 著名的學問\\
        niongnio tucike bengsen. & 超羣的本事\\
        geren ci colgoroko muten. & 出衆的能柰\\
        enduringge niyalma tutabuha tacihiyan. & 聖人所遺之訓\\
        nenehe saisai leolehe gisun. & 先賢所論之言\\
        harga\v{s}an yamun -i toktoho kooli. & 朝廷所定之例
    \end{tabularx}
\end{center}

\subsection{副動詞V-fi, V-me}
\noindent 若遇文氣難斷處,逐句只管用-fi, -me。
\begin{center}
    \begin{tabularx}{\textwidth}{XX}
        ejen oho niyalma beyebe tob obufi, harga\v{s}an yamun de enggeleme, & 人君正己以臨朝,\\
        geren hafan be tob obufi, tumen irgen be dasame, & 正羣僚以治萬民,\\
        duin mederi be tob obufi, abkai fejergi be wembuhe manggi, & 正四海以化天下,\\
        teni gurun boo be enteheme karmame, goro goidatala doro tutabuci ombi. & 方可永保國家,統垂悠久也。
    \end{tabularx}
\end{center}

\subsubsection{完成體副動詞-fi}
\noindent -fi字本是未然了,中間串語過文多。
\begin{center}
    \begin{tabularx}{\textwidth}{XX}
        boode marifi, buda jefi, tacik\={u}de jifi, saikan bengsen be tacikini. & 囘了家,喫了飯,上了學,好生學本事。\\
        ambasa saisa necin de bifi hesebun be aliyambi. & 君子居易以俟命。\\
        gisun -i onggolo yabufi, amala dahabumbi. & 先行其言,而後從之。\\
        inenggi goidafi eimeme deribufi, fe be waliyafi ice be kicembi. & 日久而生厭,舍舊而圖新。
    \end{tabularx}
\end{center}

\subsubsection{加強的完成體副動詞-pi}
\noindent 有力神情應用-pi,又此-fi字意活潑。
\begin{center}
    \begin{tabularx}{\textwidth}{XX}
        tangg\={u} halai irgen leksei wempi, tumen jaka yooni ijish\={u}n oho. & 百姓普化,萬物咸若。\\
        tumen gurun uhei h\={u}waliyapi, eiten gungge gemu badaraka. & 萬邦協和,庶績咸熙。\\
        monggon sempi harga\v{s}ambi, mujilen jempi yabumbi. & 引領而望,忍心而行。
    \end{tabularx}
\end{center}

\subsubsection{未完成體副動詞-me}
\noindent -me字在中為串貫,句尾卻又講之着。
\begin{center}
    \begin{tabularx}{\textwidth}{XX}
        wembume tuwancihiyame selgiyeme yabubumbi. & 化裁推行\\
        tacibume h\={u}wa\v{s}abume banjibume \v{s}anggabumbi. & 教養生成\\
        hiyoo\v{s}un senggime be ujeleme wesihulembi. & 敦崇孝友\\
        irgebun bithe be urebume tacimbi. & 服習詩書\\
        beyebe olho\v{s}ome tacihiyan be dahambi. & 謹身率教\\
        giyan be songkolome fafun be tuwakiyambi. & 循理奉公\\
        julgei dorolon be dahame, te -i arbun de acaname, niyalma be tuwara baita be yabure ohode, sain ak\={u} -i kooli bio? & 遵之古禮,合着時勢,待人處事,豈有不好的?
    \end{tabularx}
\end{center}

\subsubsection{“方纔、剛纔”-me saka/jaka}
\noindent “方纔、剛纔”如之何,saka, jaka上接-me。
\begin{center}
    \begin{tabularx}{\textwidth}{XX}
        yasa nicume saka, gaitai uthai hiri amgaha. & 方纔合眼,忽然就睡熟了。\\
        beye forgo\v{s}ome jaka, hercun ak\={u} geli tuheke. & 剛纔囘身,不覺又倒下了。
    \end{tabularx}
\end{center}

\subsubsection{“剛一”-me}
\noindent “纔一”、“剛一”文氣快,句中文字只需-me。
\begin{center}
    \begin{tabularx}{\textwidth}{XX}
        teni debtelin be neime, dolori acinggiyabufi sesulembi. & 纔一展卷,感切由衷。\\
        teni bithe be tuwame, alimbaharak\={u} nasame sejilembi. & 剛一臨文,曷深慨嘆。
    \end{tabularx}
\end{center}

\subsubsection{“將及”ome}
\noindent “將及”、“甫及”及“方及”,往往常有用ome。
\begin{center}
    \begin{tabularx}{\textwidth}{XX}
        juwan inenggi ome, baita wacihiyaha. & 將及旬日而蕆事。\\
        emu biya ome gungge mutebuhe. & 甫及匝月而成功。\\
        ilan aniya ome wen selgiyebuhe. & 方及三年而化成。
    \end{tabularx}
\end{center}

\subsubsection{“...即”V-me, uthai...}
\noindent -me字若連uthai,語氣緊如-mbihede。
\begin{center}
    \begin{tabularx}{\textwidth}{XX}
        emgeri sabume, uthai sambi. & 一見便知。\\
        emgeri yasalame, uthai getuken oho. & 一目了然。\\
        urunak\={u} kirimbihede, teni tusa bi. & 必有忍,乃有濟。\\
        urunak\={u} kicembihede, teni gungge bi. 必黽勉,方有功。
    \end{tabularx}
\end{center}

\subsubsection{副動詞從句V-me...V-fi...}
\noindent -me字下面有-fi字,不有-mbi即-ha, -he。
\begin{center}
    \begin{tabularx}{\textwidth}{XX}
        beyebe ilibume doro be yabume, niyaman be eldembufi gebu be algimbufi, jui oho niyalmai doro be g\={u}tuburak\={u} oci acambi.& 立身行道,顯親揚名,無虧於人子之道也可。\\
        gebu be kiceme maktacun be gaifi, deribun de kiceme duben de heoledefi, ulhiyen -i sartabure waliyabure de dosinahabi. te de acabume julge be yargiyalame, hanciki be g\={u}nime tengkicuke babe fonjime ofi, mergen de haminaha bime gosin de ibenehebi. & 沽名釣譽,勤始惰終,而日就荒廢焉。準今酌古,近思切問,而幾乎賢且進於仁矣。
    \end{tabularx}
\end{center}

\subsubsection{副動詞從句V-fi...V-me}
\noindent -fi字下面有-me字,工夫遞進語言多。
\begin{center}
    \begin{tabularx}{\textwidth}{XX}
        juse deote be tuwahai donjihai ulhiyen -i urefi, durun kemun be songkolome yabubume, inenggi goidaha manggi, g\={u}nin mujilen gulu nomhon, yabun tuwakiyan tob ujen ombi.& 使子弟見聞日熟,循蹈規矩之中,久之心地湻良,行止端重。\\
        ba tome teisu teisu ga\v{s}an dendefi, ga\v{s}an tome meni meni falga be kadalabume, hoton de oci ho\v{s}on be bodome dendefi, ga\v{s}an de oci tokso be bodome faksalafi, boo tome duka aname, ishunde serem\v{s}eme tuwakiyabu! & 每處各自分保,每保各統一甲,城以坊分,鄉以圖別,排鄰比戶,互相防閑。\\
        hergime yabure hethe ak\={u} -i urse, terei gebu be h\={u}lhafi, terei tacihiyan be efuleme, amba muru gashan sabi jobolon h\={u}turi -i jergi hacin de anagan arame, ceni holo untuhun temgetu ak\={u} -i gisun be algimbumbi. & 游食無藉之輩,陰竊其名,以壞其術,大率假災祥禍福之事,以售其誕幻無稽之談。\\
        gurun boo de fafun ilibuhangge, cohome irgen -i balai yabure be nakabufi, yarh\={u}dame sain be yabikini, miosihon be ash\={u}fi jingkini be wesihuleme tuksicuke ci jailafi elhe be bahakini serengge. & 朝廷之立法,所以禁民為非,導其為善,除邪崇正,云危就安者也。
    \end{tabularx}
\end{center}

\subsubsection{推斷原因V-fi dere/kai}
\noindent -fi連dere與kai字,暗述上文因字活。
\begin{center}
    \begin{tabularx}{\textwidth}{XX}
        niyalma teisu be dabame balai yaburengge, jurgan giyan be sarak\={u} ofi dere. & 人之越分妄為,不知夫義理耳。\\
        yaya tacin -i \v{s}anggarak\={u}ngge, cohome g\={u}nin sith\={u}rak\={u} ofi kai. & 凡學之無成者,以其心不專也。
    \end{tabularx}
\end{center}

\subsubsection{副動詞從句V-me...V-me...}
\noindent -me字平平往下串。
\begin{center}
    \begin{tabularx}{\textwidth}{XX}
        bireme yargiyalame julge be yarume, dahime dabtame ak\={u}mbume tucibume, g\={u}nin be iletu getuken obure be kicembi. & 旁徵遠引,往覆周詳,意取顯明。\\
        niyengniyeri tarime bolori bargiyame, ume erin be ufarabure. & 春耕秋敛,勿失其時。\\
        kemneme malh\={u}\v{s}ame hairame ujime, ume kemun be jurcere. 撙節愛養,無愆於度。
    \end{tabularx}
\end{center}

\subsubsection{並列的未完成體分詞V-rA...V-rA...}
\noindent 串文斷落-ra -re -ro。
\begin{center}
    \begin{tabularx}{\textwidth}{XX}
        beye selara, beyede kimcire, beyede forgo\v{s}oro, beyebe tuwacihiyara be,  & 自慊、自省、自反、自修,學者須自勉焉。\\
    \end{tabularx}
\end{center}

\subsection{表示因果關係[V-me, V-re] ohode, V-re be dahame...V-mbi}
\noindent 或-me 或-re連連用,句尾當托ohode。工夫說到效驗處,自然而然be dahame,下文還得托-mbi,起止界限方明白。
\begin{center}
    \begin{tabularx}{\textwidth}{XX}
        beyeingge be ak\={u}mbufi niyalma be giljame, sarasu de isibufi jaka be hafume, fe be urebume ice be sara, jiramin be ujeleme dorolon be fusihulere ohode, ini cisui sukdun h\={u}waliyapi, mujilen toktoro jurgan \v{s}umpi gosin urere be dahame, doro ereci wesihun erdemu ereci jiramin ombi. & 盡己而恕人,致知而格物,敦厚以崇禮,自然和氣而神凝,義精而仁熟,將見道於是乎高德於是乎厚矣。
    \end{tabularx}
\end{center}

\subsection{與現在有關的以往動作V-fi bi}
\noindent 揣度以往用-fi bi。
\begin{center}
    \begin{tabularx}{\textwidth}{XX}
        temgetu be ejefi bi. & 記上記號了。\\
        songko be sulabufi bi. & 留下蹟址了。\\
        butui erdemu iktambufi bi. & 陰隲積下了。\\
        gungge fa\v{s}\v{s}an ilibufi bi. & 功勞立下了。\\
        wen wang han, u wang han -i dasan, undehen \v{s}usihe de arafi bi. & 文武之政,布在方策。\\
        kungdz mengdz -i doro, bithe cagan de ejefi bi. & 孔孟之道,載在典籍。
    \end{tabularx}
\end{center}

\subsection{進行體V-me bi}
\noindent 正當其時-me bi多。
\begin{center}
    \begin{tabularx}{\textwidth}{XX}
        jing bithe h\={u}lame bi. & 正念著書。\\
        jing hergen arame bi. & 正寫著字。\\
        sula gisun gisuremne bisire de, sasa kaicame deribuhe. & 正在說閑話,大家嚷起來了。\\
        alban baita icihiyame bisire de, ishunde dai\v{s}ame deribuhe. & 正在辦官事,彼此鬧起來了。
    \end{tabularx}
\end{center}

\subsection{完成體的可結句構造V-hAbi}
\noindent -habi -hebi 已然語。
\begin{center}
    \begin{tabularx}{\textwidth}{XX}
        tacime bahanahabi. & 已學會了\\
        weileme \v{s}anggahabi. & 已做成了\\
        ejeme urehebi. & 已記熟了\\
        urebume dubihebi. & 已練慣了\\
    \end{tabularx}
\end{center}

\subsection{不定的未然狀態V-mbi}
\noindent 泛論未然-mbi字活。
\begin{center}
    \begin{tabularx}{\textwidth}{XX}
        erdemu be genggiyelembi. & 明德\\
        irgen be icemlembi. & 新民\\
        sain de ilinambi. & 止善\\
        ilin be sambi. & 知止\\
    \end{tabularx}
\end{center}

\subsection{完成體否定“沒、未”V-hAk\={u}}
\noindent -hak\={u}, -hek\={u}是沒、未字
\begin{center}
    \begin{tabularx}{\textwidth}{XX}
        alibume boolahak\={u}. & 未稟報\\
        donjibume wesimbuhek\={u}. & 未奏聞\\
        bahafi alahak\={u}. & 沒得告訴\\
        tucibume gisurehek\={u} & 沒說出來
    \end{tabularx}
\end{center}

\subsection{未完成體否定“不”V-rAk\={u}, V-hAk\={u}}
\noindent -rak\={u}, -hek\={u}專以不、弗說
\begin{center}
    \begin{tabularx}{\textwidth}{XX}
        dorolon -i kemnerak\={u} oci, inu yabuci ojorak\={u} kai. & 不以禮節之,亦不可行也。\\
        tacirak\={u}ci wajiha, tacici muterak\={u} oci, nakarak\={u}. & 有弗學,學之弗能,弗措也。
    \end{tabularx}
\end{center}

\subsection{狀態副動詞V-hAi}
\noindent 只管、只是、儘只語,-hai -hei -hoi字串着說。
\begin{center}
    \begin{tabularx}{\textwidth}{XX}
        yabuhai teyerak\={u} & 只管走不歇着。\\
        gisurehei nakarak\={u} & 只是說不止。\\
        bodohoi wajirak\={u} & 儘只算不完。
    \end{tabularx}
\end{center}

\subsubsection{狀態副動詞的可結句形式V-hai bi}
\noindent 情切文完意難盡,-hai, -hei, -hoi下bi連着。
\begin{center}
    \begin{tabularx}{\textwidth}{XX}
        ilihai icihiyara be aliyahai bi. & 立候辦事。\\
        bi kemuni erehei bi. & 予日望之。\\
        aifinici tulbime bodohoi bi. & 久有籌畫。
    \end{tabularx}
\end{center}

\subsection{因果關係V-me ofi}
\noindent 因為神情是ofi,上非整字必用-me。句中又作為、是講,上連整字方使得。
\begin{center}
    \begin{tabularx}{\textwidth}{XX}
        yabun tuwakiyan sain ofi, niyalma teni kunduleme tuwambi. & 因為品行好,人纔敬重。\\
        tacire de amuran ofi, bengsen teni geren ci colgorome mutehebi. & 因為好學,本事方能出衆。\\
        da dube ak\={u} ofi, tuttu mangga\v{s}ame faihacambi. & 因為無頭緒,所以為難着急。\\
        kemneme malh\={u}\v{s}ame ofi, tuttu bayan de isibuhabi. & 因為節儉,所以致富。\\
        hafan ofi baita icihiyara de, tondo mujilen tebuci acambi. & 為官蒞政,須秉公心。\\
        niyalma ofi niyalma doro be ulhirak\={u} oci, niyalma ojorongge aini. & 是人而不通人理,則何以為人。
    \end{tabularx}
\end{center}

\subsection{waka的否定用法:N waka, V-rAngge/hAngge waka}
\noindent waka之上非整字,必用-rengge與-hangge。
\begin{center}
    \begin{tabularx}{\textwidth}{XX}
        an -i ucuri ere durun -i niyalma waka, ere ainaha ni? & 平素不是這樣人,這是怎麼了?\\
        ten -i erdemu waka oci, we uttu ome mutembi? & 非至德,孰能如是乎?\\
        labdu be nem\v{s}erengge waka, jaci baitalara de acaburak\={u} ofi kai. & 不是貪多,太不符用啊。\\
        teni gebu be kicere jalin dabala, aisi be baihangge waka. & 原為圖名,非是求利益。
    \end{tabularx}
\end{center}

\subsection{名物化、話題標記V-rengge, ...}
\noindent 開口一叫使-rengge,末尾亦用長音托。
\begin{center}
    \begin{tabularx}{\textwidth}{XX}
        ambula donjime hacihiyame ejerengge, erdemu be h\={u}wa\v{s}abure doro be badaramburengge kai. & 博聞強記,所以畜德而弘業者也。\\
        fafun kooli serengge, h\={u}wangdi han sai umainaci ojorak\={u} baitalarangge. & 法例者,帝王不得已而用之也。
    \end{tabularx}
\end{center}

\subsection{因果關係V-ci banjinahangge}
\noindent 追述上文褒貶語,下用-ci banjinahangge。
\begin{center}
    \begin{tabularx}{\textwidth}{XX}
        tere tacime \v{s}anggarak\={u}ngge cohome kicerak\={u} ci banjinahangge. & 其學之不成者,乃由於不勤也。\\
        niyalma g\={u}nin -i cihai balai yaburengge, sukdun salgabun -i urhu ci banjinahangge. & 人之任意妄為,乃由氣質之偏。\\
        yaya dorolon jurgan be getukelerak\={u}ngge, bithe tacirak\={u} ci banjinahangge. & 大凡不明體義者,不肯讀書所致也。
    \end{tabularx}
\end{center}

\subsection{名物化的並列V-rA...V-rA...V-rAngge}
\noindent 三字平行歸一致,-ra -re -ro下接-rengge
\begin{center}
    \begin{tabularx}{\textwidth}{XX}
        teisu be tuwakiyara, erin de acabure, sain be sonjoro, jurgan de gurinjerengge, erdemu be wesihulere doro be sebjelere niyalma kai. & 守分、從時、擇善、徙義者,尊德樂道之人也。\\
        niyalma be warangge, ergen toodambi. bekdun be edelerengge, jiha toodambi. & 殺人的償命,欠債的還錢。\\
        sain be yabuhangge h\={u}turi bahambi. ehe be deribuhengge sui tuwambi. & 為善者享福,作惡者受罪。
    \end{tabularx}
\end{center}

\subsection{話題標記...serengge, V-hA serengge...inu/kai}
\noindent serengge是“論、夫”者字,整字與-ha -he亦可托。下文若是有、也、者,inu kai字作着落。
\begin{center}
    \begin{tabularx}{\textwidth}{XX}
        mujin serengge sukdun de dalahangge, sukdun serengge beye de jalukangge. & 夫志,氣之帥也。氣,體之充也。\\
        gosin jurgan serengge dasan -i fulehe, erun koro serengge dasan -i dube. & 夫志,理之本也。刑法者,理之末也。\\
        enduringgei tacin serengge, kungdz mengdz de badarambuhabi. doroi \v{s}o\v{s}ohon serengge, yoo han \v{s}un han ci deribuhebi. & 夫聖學,昌於鄒魯。夫道統,肇自中天。\\
        fujurungge ambasa saisa be, dubetele onggoci ojorak\={u} serengge, wesihun erdemu ten -i sain, irgen -i onggome muterak\={u} be henduhebi kai. & 有斐君子,終不可諠兮者,道盛、德至善,民之不能忘也。\\
        dulimba tob serengge fulehe inu. & 中正本也。\\
        genggiyen kengse serengge, baitalan inu. & 明斷用也。\\
        dasan serengge geli okjiha ulh\={u} -i adali kai. & 夫政也者,蒲蘆也。
    \end{tabularx}
\end{center}

\subsection{倒裝...ofi kai}
\noindent ofi kai是倒裝語。
\begin{center}
    \begin{tabularx}{\textwidth}{XX}
        jalan -i niyalma karman tuwakiyan be daruhai k\={u}bulirengge, terei jurgan aisi -i ilgabun be sark\={u} ofi kai. & 世人之常變操守者,不知夫義利之區也。\\
        yaya niyalma da banin be ufarabuhangge, sain be genggiyelefi, tuktan de dah\={u}me muterak\={u} ofi kai. &凡人之失其本性者,不能明善復初也。
    \end{tabularx}
\end{center}

\subsection{倒裝...de kai, be kai}
\noindent de kai, be kai是也說。
\begin{center}
    \begin{tabularx}{\textwidth}{XX}
        niyengniyeri edun -i selgiyere, juwari aga -i simeburengge, abkai erin de kai. & 春鼓之以風,夏潤之以雨,是天之時也。\\
        nuhu olhon ba ire fisihe de acara, nuhaliyan durbehun(derbehun) ba handu jeku de acarangge. na -i aisi de kai. & 高躁者宜黍稷,下濕者宜秔稻,是地之利也。\\
        mujin be toktobumbi serengge, emu mujilen -i sain be sonjofi teng seme tuwakiyara be kai. jibsen be da arambi serengge, beye mujilen be jafatame bargiyatafi g\={u}wabsi ak\={u} obure be kai. & 所謂定志者,是一心擇善而固執之也,主敬者,是攝束身心而不他適也。
    \end{tabularx}
\end{center}

\subsection{引經據典...de henduhengge}
\noindent 若是引經與據典,必用de henduhengge。句尾須用sehebi,起止界限方明白。
\begin{center}
    \begin{tabularx}{\textwidth}{XX}
        %44
        dorolon -i nomun de henduhengge, dorolon hafuci teisu toktombi sehebi. ede wesihun fusih\={u}n be ilgara, dergi fejergi be faksalara de, dorolon ci dulenderengge ak\={u} be seci ombi. & 《禮記》曰,禮達而分定,可見辨尊卑。分上下,莫過於禮也。\\
        hiyoo\v{s}ungga nomun de henduhengge, dergi be elhe obure, irgen be dasara de, dorolon ci sain ningge ak\={u} sehebi. erebe tuwahade dorolon serengge, jalan de aisilara irgen de dalara oyonggo baita kai. & 《孝經》云,安上治民,莫善於禮,是知禮也者,輔世長民之要務也。
    \end{tabularx}
\end{center}

\subsection{引經據典...henduhe, ...(sefi)...sehengge}
\noindent 用henduhe引述起,句下須用sehengge。往往中間用sefi,只看單連意何如。
\begin{center}
    \begin{tabularx}{\textwidth}{XX}
        julgei niyalma henduhe, belheme jabduci, jobocun ak\={u} ombi sehengge. & 昔人云,有備無患。\\
        yaya baita be doigom\v{s}oci ilinara, doigom\v{s}orak\={u} oci waliyabure be henduhebi. & 言凡事豫則立,不豫則廢也。\\
        kungdz -i henduhe, julgei niyalma gisun tucirak\={u}ngge, beyei haminarak\={u} jalin girume oci kai sehengge. & 孔子云,古者言之不出,恥躬之不逮也者。\\
        cohome untuhun gisun tucibure gojime, beyei yabume yargiyan -i songkolome muterak\={u} ayoo sere turgun kai. & 蓋恐其徒托空言,不能躬行實踐也。\\
        fudz -i henduhe, tacimbime erinderi urebu sefi, geli tacire de amcarak\={u} -i adali oso sehengge, cohome niyalma be majige andande seme nakaci ojorak\={u} erin be amcame kiceme tacikini sere g\={u}nin kai. & 子曰學而時習之,又曰學如不及,乃使人及時勤學不可須臾而或輟也。\\
    \end{tabularx}
\end{center}

\subsection{“...是謂...”...be...sembi}
\noindent 為字謂字是sembi,上面必有be叫着。
\begin{center}
    \begin{tabularx}{\textwidth}{XX}
        umesi ten tumen jaka be elbehengge be, abka sembi. umesi jiramin tumen jaka be alihangge be, na sembi. inenggi abka na de elderengge be \v{s}un sembi. dobori abka na de elderengge be, biya sembi.&至高而覆萬物者,為天。至厚而載萬物者,為地。日閒光照於天地者,曰日。夜裏光照於天地者,曰月。\\
        na de cokcohon -i bisirengge, alin sembi. \v{s}eri deri eyeme tucinjirengge be, muke sembi. abkai hesebuhengge be, banin sembi. sukdun -i salgabuhangge be buyenin sembi. turgun bifi yaburengge be, aisi sembi. turgun ak\={u} yaburengge be, jurgan sembi.& 在地上高聳的,叫作山。由泉中流出者,叫作水。天命之謂性,氣禀之謂情。有所為而為,謂之利。無所為而為,謂之義。
    \end{tabularx}
\end{center}

\subsection{引經據典...sehengge}
\noindent 代申其意sehengge,發明已往經典多。
\begin{center}
    \begin{tabularx}{\textwidth}{XX}
        irgebun -i nomun de, \v{s}e deyeme abka de isinambi, nimaha tunggu de godombi sehengge, tere dergi fejergi de iletulehe be henduhebi. & 《詩》云,鳶飛戾天,魚躍於淵,言其上下察也。\\
        dasan -i nomun de, gungge be wesihun oburengge, damu mujin de, doro be badaramburengge, damu kicebe de sehengge. cohome baita mujin daci ishunde nikendufi mutebumbi sere turgun. & 《書》曰,功崇惟志,業廣惟勤,蓋業與志本相須而成也。\\
        jijungge nomun de, ini erdemu entehen ak\={u} de, eici girucun be alimbi sehengge, ere gisun absi sain.& 《易》曰,不恆其德,或承之羞,旨哉言乎。\\
        dorolon -i nomun de, mafa be wesihuleme ofi, tuttu da be ginggulembi, da be gingguleme ofi, tuttu muk\={u}n be bargiyambi sehengge, niyalma doro de urunak\={u} muk\={u}n be h\={u}waliyambure be ujen obuci acara be getukelehengge. & 《禮》云,尊祖,故敬宗,敬宗,故收族。明人道必以睦,族為重也。\\
        h\={u}lha holo be geterembre de mangga sehengge ede kai. & 此所謂盜賊難弭也。 \\
        coohai doro be ga\v{s}an falga banjibure de baktambuhabi sehengge, erebe kai. & 所謂寓兵法於保甲中也。
    \end{tabularx}
\end{center}

\subsection{“想必”ainci...(dere/aise)}
\noindent ainci下用dere aise,原是猜疑話來着。如今竟作蓋字用,dere aise不必托。
\begin{center}
    \begin{tabularx}{\textwidth}{XX}
        tere baita ainci \v{s}anggaha dere. & 那件事情想是成了罷。\\
        ere jaka ainci efujehe dere. & 這宗東西想是壞了罷。\\
        ainci hiri amgaha aise. & 想必睡着了罷。\\
        ainci hono getehek\={u} aise. & 想必還沒醒罷。\\
        ainci julgei ebsi irgen -i an kooli, gemu kicebe malh\={u}n be wesihun obuhabi. & 蓋自古民風,皆貴乎節儉。\\
        ainci dorolon serengge, abka na -i enteheme, tumen jaka -i ilgabun. & 蓋禮為天地之經,萬物之序。
    \end{tabularx}
\end{center}

\subsection{表示推斷的語氣助詞...dere, ...V-mbi dere}
\noindent dere又作罷而已,如用dabala一樣托。上面若非遇整字,非用-mbi不可托。
\begin{center}
    \begin{tabularx}{\textwidth}{XX}
        deribun bisire duben bisirengge, teni niyalma seci ombi dere. & 有始有終的,纔是人罷。\\
        hehe niyalma bikai? uyun niyalmai teile dere。& 有婦人嗎?九人而已。\\
    \end{tabularx}
\end{center}

\subsection{“又”、“而且”...bime, geli...,...dade, geli...}
\noindent bime本作而且用,整單破連用法活。下有又字加geli,若遇已然即-ha, -he。dade也是又字意,geli之字緊連著。
\begin{center}
    \begin{tabularx}{\textwidth}{XX}
        gosin jurgan bime tondo nomhon. erdemu be iktambumbime geli gosin be yabumbi. & 仁義而且忠厚,積德而且累仁。\\
        \v{s}ayo \v{s}ayolambime geli nomun jondombi. & 持齋而且念經。\\
        bithe be h\={u}laha bime hergen be inu araha. &讀了書而且把字亦寫了。\\
        niyalma de tusa arara dade, geli jaka be aisi obuhabi. & 濟人而且利物。\\
        goosin de yendenure dade, geli anah\={u}njara be kicembi. & 興仁而且講讓。
    \end{tabularx}
\end{center}

\subsection{“而且”ere/tere dade..., ede bime...}
\noindent 上文頓住又而且,ere dade與tere dade。ede bime亦如是,緊接上文字句說。
\begin{center}
    \begin{tabularx}{\textwidth}{XX}
        ere dade minggan hacin -i k\={u}bulire, tumen hacin -i forgo\v{s}orongge, emu mujilen de fakjilambi. & 而且千變萬化,宰於一心。\\
        tere dade aniya h\={u}sime bithe h\={u}larak\={u} oci okini, emu inenggi seme buya niyalma de hanci oci ojorak\={u}. & 而且甯可終歲不讀書,不可一日近小人。\\
        ede bime ten -i unenggi -i doro, an dulimba ci tucinerak\={u}. & 而且至誠之道,不外中庸。
    \end{tabularx}
\end{center}

\subsection{“然而”uttu bime}
\noindent uttu bime然而用。
\begin{center}
    \begin{tabularx}{\textwidth}{XX}
        uttu bime han ojorak\={u}ngge ak\={u} kai. & 然而不王者,未之有也。
    \end{tabularx}
\end{center}

\subsection{“乃、且”tuttu bime}
\noindent tuttu bime“乃、且”多。
\begin{center}
    \begin{tabularx}{\textwidth}{XX}
        tuttu bime ulin jakai mangga be g\={u}nirak\={u}, g\={u}nin -i cihai mamgiyame kobcihiyadambi. & 乃不知物力艱難,任意奢侈。\\
        tuttu bime ambasa saisa be tetu\v{s}eci ojorak\={u}. & 且君子不器。
    \end{tabularx}
\end{center}

\subsection{“A而不B”...gojime, ...dabala...V-rak\={u}}
\noindent 僅能如彼不能此,上好下歹是gojime。變文上下皆好意,dabala句下-rak\={u}托。句中皆以而字用,只看本文意思說。
\begin{center}
    \begin{tabularx}{\textwidth}{XX}
        gebu be kicere gojime, yargiyan be bairak\={u}. & 務名而不求實。\\
        aisi be yendebure gojime, jemden be geteremburak\={u}. &興利而不除弊。\\
        jurgan be tob obure dabala, aisi be kicerak\={u}. & 正其誼而不謀其利。\\
        doro be genggiyelere dabala, gungge be bodorak\={u}. & 明其道而不計(功)。
    \end{tabularx}
\end{center}

\subsection{伴隨副動詞...rAlame...}
\noindent 正然如此又如彼,-ra- -re-等字連-lame。句中作為隨且用,行文用意甚活潑。
\begin{center}
    \begin{tabularx}{\textwidth}{XX}
        h\={u}laralame arambi. & 隨念隨寫\\
        yaburelame tuwambi. & 隨走隨看\\
        sara\v{s}aralame feliyembi. & 且逛且遊\\
        gisurerelame injembi. & 且説且笑
    \end{tabularx}
\end{center}

\subsection{“如果、倘若”aika/yala/unenggi/aikabade...oci/ohode/V-ci}
\noindent aika、yala、unenggi,下非oci即ohode。aikabade亦如此,設若如果倘或說。若是句短文氣近,亦可用-ci字托着。
\begin{center}
    \begin{tabularx}{\textwidth}{XX}
        aika tacirak\={u} oci, adarame giyan be same mutembini? & 設若不學,豈能知理?\\
        yala mujangga oci, fafun hergin be ambula necihe be dahame, ek\v{s}eme fere heceme sihame beideci acambi. & 如果屬實大干法紀,亟宜澈底根究。\\
        unenggi baita tome jurgan -i lashalame mutere ohode, ini cisui dabatala ebele -i calabun ak\={u} ombi. & 果能事事斷之以義,自無過不及之差矣。\\
        aikabade jug\={u}n -i andala aldasilara oci, inu hairakan ak\={u} semeo? & 倘或半途而廢,不亦惜乎?\\
        unenggi beyebe tob obuci, dasan de danara de ai ojorak\={u}? &苟其正身,於徔政乎何有?
    \end{tabularx}
\end{center}

\subsection{表示猜測、“難道”的aika...gese, aika...bio/nio?}
\noindent aika又有兩樣講,猜度神情難道說。猜度下用gese字,難道bio、nio等字托。
\begin{center}
    \begin{tabularx}{\textwidth}{XX}
        ere baita be aika icihiyame muterak\={u} -i gese. & 這事兒想必是不能辦。\\
        tere jug\={u}n be aika kiceci ojorak\={u} -i gese. & 那道兒莫非是不可謀。\\
        aika baita bio? &難道有事麼?\\
        aika banjinarak\={u} nio? &難道不行麼?\\
        aika sain ak\={u} semeo? &難道不好麽?\\
        aika mejige ak\={u} mujanggao? & 難道沒信麼?
    \end{tabularx}
\end{center}

\subsection{“萬一”talude...sehede}
\noindent talude“萬一”、“偶”、“或”用,比語設或sehede。
\begin{center}
    \begin{tabularx}{\textwidth}{XX}
        talude ufaraha sehede, aliyaha seme amcaburak\={u} kai. & 設或萬一丟了呢?後悔不及呀。\\
        talude erun de tu\v{s}aha de, juse sargan sui tuwambi. & 設使偶罹於法,則累及妻孥。\\
        te bici, fafun -i asu de emgeri fehunehe sehede, tangg\={u} hacin -i gosihon be alimbi. & 試思一蹈法網,百苦備嘗。
    \end{tabularx}
\end{center}

\subsection{“何必”V-fi ainambi, N be ainambi}
\noindent ainambi上破字用-fi,整字過文須用be。
\begin{center}
    \begin{tabularx}{\textwidth}{XX}
        g\={u}wabsi baifi ainambi. & 何必他求?\\
        goromime seolefi ainambi. &何須遠慮?\\
        jiha fulu be ainambi. &何用錢多?\\
        dorolon largin be ainambi. &何必禮大?
    \end{tabularx}
\end{center}


\subsection{“須”V-ci acambi, “合乎”、“符合”N de acanambi}
\noindent acambi上必接-ci,acanambi上應用de。
\begin{center}
    \begin{tabularx}{\textwidth}{XX}
        sain be yabuci acambi. &須當為善。\\
        ehe be yabuci acarak\={u}. & 不應作惡。\\
        julge de acanambi,te de acanarak\={u}. & 合乎古,不合乎今。
    \end{tabularx}
\end{center}

\subsection{“克、能、會”[V-me, N be] mutembi/bahanambi}
\noindent mutembi與bahanambi,整字接be破用-me。
\begin{center}
    \begin{tabularx}{\textwidth}{XX}
        ganggan be mutembi. genggen be muterak\={u}. &克其剛,不克其柔。\\
        ambula isibume mutembi. geren de tusa arame muterak\={u}. & 能博施,不能濟衆。\\
        bithei tacin be bahanambi. cooha bodogon be bahanarak\={u}. &會文學,不會武畧。\\
        gurun be dasame bahanambi. cooha be baitalame bahanarak\={u}. & 善治國,不善用兵。
    \end{tabularx}
\end{center}

\subsection{“如何/how is” V-ci antaka}
\noindent antaka上用ci字,整字長音也使得。
\begin{center}
    \begin{tabularx}{\textwidth}{XX}
        \v{s}ayo \v{s}ayolarangge, nomun nomulara ci antaka? & 持齋比說法如何?\\
        samdi tere anggala, doro be ulhici antaka? & 坐禪何如悟道?\\
        mini gisurehe gisun antaka? &我說的話如何?\\
        mini niruha nirugan antaka? &他畫的畫如何?\\
        fudz -i kemuni gisurerengge antaka? &子所雅言者何如?\\
        fudz -i asuru gisurerak\={u}ngge antaka? &夫子罕言者何如?
    \end{tabularx}
\end{center}

\subsection{將要、臨了[V-hA, V-rA] nam\v{s}an}
\noindent nam\v{s}an上接-ha -he等,亦可上用-ra -re -ro。
\begin{center}
    \begin{tabularx}{\textwidth}{XX}
        tu\v{s}an ci aljaha nam\v{s}an, neneme getukeleme joolame afabuha. & 臨卸任,先交代明白了。\\
        simneme genehe nam\v{s}an, uthai dosiki seme erehunjehe. & 將赴考,就指望着中。\\
        jurare nam\v{s}an, kunesun werihe. & 臨起身,留下盤費了。\\
        marire nam\v{s}an, aciha gajiha. & 臨回頭,帶來行李了。
    \end{tabularx}
\end{center}

\subsection{原來dade/原先daci/幾乎elekei...[V-hA, bihe]}
\noindent dade、daci、elekei下,bihe與-ha、-he等字托。
\begin{center}
    \begin{tabularx}{\textwidth}{XX}
        dade g\={u}wa bade tembihe, dade emu bade banjimbihe. & 原先在別處住,起初仝居。\\
        daci jihengge goidaha, dac hihalarak\={u} bihe. & 由來尚矣,從不稀罕。\\
        elekei lahin daha, elekei turibuhe. & 幾乎受累,差一點兒失落了。
    \end{tabularx}
\end{center}

\subsection{“...後(又)”-hA manggi...(geli)...}
\noindent manggi本講虛時候,繙作“而後”、“然後”說。-ka -ha -ko -ho與-ke -he,字句之上緊連著。上接口氣作“既”講,緊接下句geli托。
\begin{center}
    \begin{tabularx}{\textwidth}{XX}
        gebu algika manggi, absi derengge. gung ilibuha manggi, absi horonggo. & 享了名的時候,多體面。立了功的時候,多威武。\\
        jurgan \v{s}ungge manggi, gosin teni urembi. & 義精,而後仁熟。\\
        wen selgiyehe manggi, tacin teni sain ombi. & 化行,而後俗美。\\
        edun toroko manggi, jai geneki. & 風定了,然後再去。\\
        \v{s}un forgo\v{s}oho manggi, jai guriki. & 日頭轉過去,然後再挪。\\
        erebe bahana manggi, geli terebe tacimbi. & 會了這箇,又學那箇。\\
        ede kice manggi, geli tede fa\v{s}\v{s}ambi. & 勤於此,又勉於彼。
    \end{tabularx}
\end{center}

\subsection{泛論某一個動作V-rA de}
\noindent 泛論常文de字過,-ra -re等字上連着。
\begin{center}
    \begin{tabularx}{\textwidth}{XX}
        beyebe ilibure de unenggi be da obumbi. & 立身以誠為本。\\
        beyebe tuwakiyara de ginggun be nenden obumbi. & 持躬以敬為先
    \end{tabularx}
\end{center}

\subsection{“突然、倏忽”[V-rA, N-i] andande, “之間、之際”[V-rA, N-i] sidende}
\noindent andande上接整字,sidende、-i字與-ra、-re、ro。
\begin{center}
    \begin{tabularx}{\textwidth}{XX}
        ilihai andande, belheme jabduha. &立時之間,豫備妥當。\\
        gaitai andande, gaijara waliyarangge meimeni encu. & 頃刻之間,取舍各殊。\\
        dartai andande, k\={u}bulire arbun hacinggai tucinjimbi. &倏忽之間,變態百出。\\
        majige andande, tulhun tugi duin ici dekdehe. & 須臾之間,陰雲四起。\\
        emu cimari andande gaihari ulhifi, kenehunjere g\={u}nin subuhebi. & 一朝猛省,疑團解釋。\\
        emu cimari andande aliyame halame, juhei adali wembi, talman -i adali samsimbi. & 一旦悔改,如冰消霧釋。\\
        abka na -i sidende, niyalma umesi wesihun. &天地之閒,人為至貴。\\
        aga agara sidende, kiceme baitalarangge jing teisu. & 下雨之際,正好用工。\\
        sula leolere sidende, gebu tacihiyan de holbobuhabi. & 閒談之中,有關名教。\\
        acabufi bodoro sidende, majige farfabuci ojorak\={u}. & 會計之頃,毋少紊亂。
    \end{tabularx}
\end{center}

\subsection{“之間”[V-rA/hA, N-i] nerginde}
\noindent nerginde上接-i,與-ra、-re、-ha、-he等字多。
\begin{center}
    \begin{tabularx}{\textwidth}{XX}
        bekte bakta -i nerginde, ejehengge getuken ak\={u}. & 倉促之閒,記憶不清。\\
        tolome baicara nerginde, gaitai ice hafan halaci ojorak\={u}. & 盤查之際,不可頓易生手。\\
        seoleme g\={u}nire nerginde, lak seme merkime baha. & 尋思之下,驀然想起。\\
        nimeku manggalaha nerginde, hercun ak\={u} wangga \v{s}angga oho. & 病篤之餘,不覺昏沈。\\
        yoo fusejehe nerginde, encu hacin -i fintame nimembihe. & 癰潰之頃,異常痛楚。
    \end{tabularx}
\end{center}

\subsection{“既然、既是”V-ci tetendere, V-hA/rA be dahame}
\noindent tetendere上用ci,dahame上必須be。ci tetendere為既是,be dahame作既然說。be dahame上還有字,若非ak\={u}尾即-ha -re。
\begin{center}
    \begin{tabularx}{\textwidth}{XX}
        alban game yabuci tetendere, uthai kiceme fa\v{s}\v{s}aci acambi. &既是當差,就該黽勉。\\
        tacik\={u}de jihe be dahame, ainu kicen baitalarak\={u} ni? &既然上學來了,為何不用工呢?\\
        bithe h\={u}lame muterak\={u} be dahame, geli adarame \v{s}u fiyelen be arame mutembini? & 既然不能讀書,又焉能作文章呢?\\
        tenteke baita yabure be dahame, sain niyalma seci ombi. & 既然似此行事,可謂善人。
    \end{tabularx}
\end{center}

\subsection{“已經”emgeri}
\noindent “已然”、“業經”是emgeri,下用-ha、-he等字托。
\begin{center}
    \begin{tabularx}{\textwidth}{XX}
        emgeri icihiyaha, aiseme ek\v{s}embi? & 已然辦了,何必忙?\\
        emgeri gisurehe, hono aifumbio? & 已然說了,還改口麼?\\
        emgeri onggoho, aiseme jonofi ainambi? &業經忘了,何必提他?
    \end{tabularx}
\end{center}

\subsection{未先onggolo, 尚未unde}
\noindent onggolo未先,unde尚未,-ra、-re、-ro字上連着。
\begin{center}
    \begin{tabularx}{\textwidth}{XX}
        sindara onggolo emgeri bodome jabduha. & 未放之先,已算妥了。\\
        wesire onggolo, uthai sereme ulhihe. & 未陞之先,就覺悟了。\\
        forgo\v{s}oro onggolo, neneme tomilame toktoho. & 未調之先,先擬定了。\\
        tomilara unde de, neneme mejige baha. &尚未派,先得了信了。\\
        afabure unde de, emgeri getukeleme tuwaha. & 尚未交,已看明白了。\\
        toktoro unde de, aifinici hebe\v{s}ehe. & 尚未定,早就商量了。
    \end{tabularx}
\end{center}

\subsection{未完V-nggAlA}
\noindent 若是未然文氣快,接以-nggala與-nggele。
\begin{center}
    \begin{tabularx}{\textwidth}{XX}
        gisun wajinggala, uthai genehe. &話未完,便去了。\\
        baita tucinjinggele, neneme jailaha. &事未發,先躲了。
    \end{tabularx}
\end{center}

\subsection{“如果做...”uttu ohode}
\noindent 由工致效推開講,須用uttu ohode。
\begin{center}
    \begin{tabularx}{\textwidth}{XX}
        uhei beye be ginggulere baitalan be kemnere, sain irgen ojoro be kice. weihuke oilohon tem\v{s}ere dab\v{s}ara ehe tacin be wacihiyame waliya. uttu oohode, geren -i tacin gulu jiramin, boo tome h\={u}waliyasun necin ojoro be dahame, gurun booi erdemu wen -i \v{s}anggan be urgunjeme harga\v{s}aci ombi kai.& 共勉為謹身節用之庶人,盡除夫浮薄囂凌之陋習。則風俗醕厚,家室和平,而朝廷德化之成可以樂觀也。\\
        jase be tosoro de, haksan oyonggo be saci acambi. mederi be serem\v{s}ere de, edun boljon be ulhici acambi. uttu ohode, meimeni baita de teni g\={u}tucun ak\={u} ombi. & 備邊,則險要之宜知。防海,則風濤之宜悉。庶幾無負本業矣。
    \end{tabularx}
\end{center}

\subsection{“如之何、在怎麼情況下”adarame ohode}
\noindent 不得主意問所以,是adarame ohode。
\begin{center}
    \begin{tabularx}{\textwidth}{XX}
        adarame ohode, sain. & 怎麼着好?\\
        adarame ohode, ombi. & 如之何則可?
    \end{tabularx}
\end{center}

\subsection{“如何、怎麼/how to”adarame V-rA}
\noindent 著急無法如之何,adarame下用-ra、-re、ro。
\begin{center}
    \begin{tabularx}{\textwidth}{XX}
        adarame gamara? &怎麼處?\\
        adarame weilere? & 怎麼作?\\
        adarame bodoro? &那麼算。
    \end{tabularx}
\end{center}

\section{dasame foloho manju gisun -i untuhun hergen -i temgetu jorin bithe\\重刻清文虛字指南編 下\\\large{蒙古\;萬福厚田\;著\\
漢軍\;鳳山禹門\;訂}}

\subsection{“不知...?”maka}
\noindent 疑問不知是maka,原來敢則是dule。二字之下一義用,若非-mbi即ni、nio托。
\begin{center}
    \begin{tabularx}{\textwidth}{XX}
        maka udu bi? &不知幾許\\
        maka adarame gamambi. &不知怎處\\
        maka ainambini. & 不知作什麼\\
        maka aisembini. & 不知說什麼\\
        maka binio. & 不識有諸\\
        maka mujangga nio. &不識果否\\
        dule sembi. & 原來知道\\
        dule ombi. &敢則可以\\
        dule \v{s}a\v{s}un ak\={u} ni. & 原來不齊全\\
        dule teisu ak\={u} ni. & 敢則不相稱\\
        dule uttu nio. & 原來這樣嗎\\
        dule ak\={u} nio. & 敢則沒有嗎
    \end{tabularx}
\end{center}

\subsection{“想必”V-hA/hAk\={u}/rak\={u} aise}
\noindent 追憶想是用aise,上連整字與-hA、-he、-hak\={u}、-hek\={u}、-rak\={u}等,還有-mbi與bi字多。
\begin{center}
    \begin{tabularx}{\textwidth}{XX}
        \v{s}anggaha baita aise. & 想是成案\\
        toktoho gisun aise. & 想是成語\\
        neneme yoha aise. &想是先走了罷\\
        teni jihe aise. & 想是纔來了罷\\
        hono isinjihak\={u} aise. & 想必還未到\\
        geli genehek\={u} aise. &想必又沒去\\
        umai sark\={u} aise. & 想必並不知\\
        yala ak\={u} aise. &想必真沒有\\
        hergen arambi aise. & 想是寫字呢\\
        beri tatambi aise. & 想是拉弓呢\\
        saligan bi aise. &想必有主宰\\
        boode bi aise. &想必在家裏
    \end{tabularx}
\end{center}

\subsection{“不過、無非”manggai...dabala/dere}
\noindent 不過無非是manggai,下托dabala與dere。
\begin{center}
    \begin{tabularx}{\textwidth}{XX}
        manggai untuhuri gisurere dabala. & 不過空說\\
        manggai baibi jonoro dabala. &不過白提\\
        manggai ainame gamambi dere. & 無非塞責\\
        manggai ton arambi dere. & 無非充數
    \end{tabularx}
\end{center}

\subsection{提示話題、接續上下文seci}
\noindent 自說自解用seci,上輕下重襯文活。
\begin{center}
    \begin{tabularx}{\textwidth}{XX}
        beyebe ilibure de unenggi be fulehe obumbi seci, ele ginggun be nenden obuci acambi. & 自立以誠為本,尤須以敬為先。\\
        jili de kimun banjinaha seci, kimun de jili ele nemebumbi.& 忿以成讐,讐而益忿。
    \end{tabularx}
\end{center}

\subsection{引述sehe}
\noindent 述說人言使sehe,用be tuwaci串下說。
\begin{center}
    \begin{tabularx}{\textwidth}{XX}
        ini gisun, si bi imbe gisurehe seme henduhe sehe. & 他說你說我說了他了\\
        weri simbe uttu tuttu sehe, inuo wakao? & 別人說你如此如彼,然乎否乎?\\
        hendure balama, damu sain baita be yabu, karulame acabure be ume fonjire sehe. & 常言道,但行好事,莫問前程。\\
        dekdeni henduhengge, bithe coohai erdemu be tacime bahafi, h\={u}wangdi han -i boode baitalabumbi sehe. &諺云,學成文武藝,貨於帝王家。\\
        dekdeni gisun, dabagan deri dabambi, lifagan deri lifambi sehe. &諺云,近硃者赤,近墨者黑。\\
        fudz -i henduhengge, tondo akdun be da obu sehe be tuwaci, jalan de acabure oyonggo, tondo akdun ci tucinarak\={u} be seci ombi.& 子曰,主忠信,可見涉世之要,不外乎忠信也。
    \end{tabularx}
\end{center}

\subsection{轉此成彼sehei}
\noindent 轉此成彼是sehei,只顧之意暗含着。
\begin{center}
    \begin{tabularx}{\textwidth}{XX}
        murtash\={u}n be tuwancihiyambi sehei, tob be ufaraha. &矯枉而失正\\
        baita dekdebufi niyalma be tuhebuki sehei, beye elemangga weile de tuhenembi. &造釁以傾人,究之布井以自陷。\\
        alban kame yabumbi sehei, cisu baita be gemu sartabuha. & 只顧當差,把私事都耽誤了。\\
        ere emu hacin be taciki sehe, g\={u}wa be yooni darak\={u} oho. & 只顧學這一樣兒,別的全不管了。
    \end{tabularx}
\end{center}

\subsection{命令式obu, oso}
\noindent 使令人辭硬口氣,整字obu與oso。
\begin{center}
    \begin{tabularx}{\textwidth}{XX}
        niyalma seme jalan de banjifi, niyaman be uilere de urunak\={u} hiyoo\v{s}ula, dergi be uilere de, urunak\={u} ginggule. & 人生在世,事親必孝,事上必敬。\\
        geren be tuwara de, urunak\={u} mujilen be kumdu obu. & 待眾必虛心\\
        beyebe tuwakiyara de, urunak\={u} hanja gingge oso. uttu oci teni niyalma seci ombi kai. & 處己必廉潔,方可謂之人也已。
    \end{tabularx}
\end{center}

\subsection{命令式 + sembi}
\noindent 半截字下接sembi,亦作使令口氣說。
\begin{center}
    \begin{tabularx}{\textwidth}{XX}
        imbe aibe yabu sembi? & 令他作什麼\\
        mimbe aibe gisure sembi? & 叫我說什麼?\\
        simbe bithe h\={u}la serengge, simbe sain be tacikini serengge kai. &教你念書,是教你學好啊。
    \end{tabularx}
\end{center}
%64

\subsection{形容詞後綴-saka}
\noindent 整字下截接-saka,助語神情重字多。
\begin{center}
    \begin{tabularx}{\textwidth}{XX}
        iletusaka holtoho kai, hojokesaka ere absi. & 明明的是撒謊,好好的是怎說。
    \end{tabularx}
\end{center}

\subsection{祈使態-ki, -ki...bai/dere}
\noindent -ki字本作罷、請講,bai與dere也托得。
\begin{center}
    \begin{tabularx}{\textwidth}{XX}
        si juleri yarh\={u}daki. & 你在前導引罷\\
        bi taka teyeki. & 我暫且歇歇罷\\
        wesifi teki. & 請陞上去坐\\
        juleri yoki. & 請在頭裏走\\
        amasi bedereki bai. & 請回去罷\\
        ak\={u} obuki bai. & 請免了罷\\
        erdeken -i belheki dere. & 早些預備罷\\
        doigom\v{s}ome jebkeleki dere. & 豫先隄防罷
    \end{tabularx}
\end{center}

\subsection{“想要”-ki sembi}
\noindent 欲要將是-ki sembi。
\begin{center}
    \begin{tabularx}{\textwidth}{XX}
        niyalmai mujilen be tuwancihiyaki sembi.& 欲正人心\\
        geren -i tacin be jiramileki sembi. & 欲厚風俗\\
        ambasa saisa oki sembi. & 要作君子\\
        buye niyalma oki sembi. & 要作小人\\
        amba doro be mutebuki sembi. & 大道將成\\
        ajige endebuku be fuliyambuki sembi. & 小過將赦
    \end{tabularx}
\end{center}

\subsection{懇請的祈使-rAo}
\noindent -rao、-reo懇求祈請說。
\begin{center}
    \begin{tabularx}{\textwidth}{XX}
        kesi isibume \v{s}olo \v{s}angnarao. & 懇恩賞假\\
        yooni obume giljarao. & 求賜矜全\\
        k\={u}bulime hafumbureo. & 祈為變通\\
        da dube be fisembureo. & 請述原委\\
        sai\v{s}ame huwekiyebureo. & 懇請獎勵\\
        baktambume gamarao. & 求為包涵
    \end{tabularx}
\end{center}

\subsection{“歸於、屬於”de ombi}
\noindent 歸於、屬於de ombi。
\begin{center}
    \begin{tabularx}{\textwidth}{XX}
        ere gese yabun, dubentele dursuki ak\={u} de ombi. & 似此為人,終久歸於不肖。\\
        bithei tacin -i amba \v{s}anggan, fudz de ohobi. & 儒者之大成,屬於夫子。
    \end{tabularx}
\end{center}

\subsection{“作為、列為”de obu}
\noindent de obu作為、列為說。
\begin{center}
    \begin{tabularx}{\textwidth}{XX}
        tesu idui oron be, neneme baitalara de obu. & 本班之缺,作為儘先。\\
        gemun -i hafasai simnen simnere de uju jergi de obuha. & 京察列為一等。
    \end{tabularx}
\end{center}

\subsection{“庶乎、幾乎”haminambi}
\noindent 庶乎、幾乎haminambi。上破接-me整用de。
\begin{center}
    \begin{tabularx}{\textwidth}{XX}
        calabun ak\={u} ome haminambi. & 庶乎不差\\
        umesi taifin de haminambi. & 幾乎至治
    \end{tabularx}
\end{center}

\subsection{“將及、將及”V-me/re hamime}
\noindent 將及、將及hamime,字上非-me必有-re。
\begin{center}
    \begin{tabularx}{\textwidth}{XX}
        inenggi dulin ome hamime, ser seme edun daha. & 將近晌午,微風颯颯。\\
        gerhen mukiyeme hamime, feser seme agame turibuhe. & 將近黃昏,細雨紛紛。\\
        gurun boo yendeme hamime, urunak\={u} sain sabi bi. & 國家將興,必有貞祥。\\
        gungge fa\v{s}\v{s}an mutebure hamime, ainaha seme ja obufi tuwaci ojorak\={u}. & 功業及成,慎勿輕視。
    \end{tabularx}
\end{center}

\subsection{“將近”hamika, isika}
\noindent hamika與isika,用法如同hamime。
\begin{center}
    \begin{tabularx}{\textwidth}{XX}
        inenggish\={u}n ome hamika. & 將近傍午\\
        yamjish\={u}n ome hamika. & 將近日暮\\
        amba arbun \v{s}anggara hamika. & 大局垂成\\
        weilen -i h\={u}sun wajira isika. & 工作迨竣
    \end{tabularx}
\end{center}

\subsection{“喜好”...de amuran, “小心”...de guwelke}
\noindent amuran與guwelke,二字之上緊連de。
\begin{center}
    \begin{tabularx}{\textwidth}{XX}
        faksi be baire de amuran.&好取巧\\
        cisu be yabure de amuran. &好營私\\
        ufarara de guwelke. &小心失了\\
        efujere de guwelke. & 仔細壞了
    \end{tabularx}
\end{center}

\subsection{“不知”sark\={u}, “不必”baiburak\={u}, “不須”joo}
\noindent sark\={u}, baiburak\={u}, joo,三字上靣得接be。
\begin{center}
    \begin{tabularx}{\textwidth}{XX}
        inu waka be sark\={u}.&不知是否\\
        ojoro ojorak\={u} be sark\={u}. &不知可否\\
        faihacara be baiburak\={u}. &不必着急\\
        mangga\v{s}ara be baiburak\={u}. &不用為難\\
        gemun de jidere be joo. &毋庸來京\\
        jase de tebunere be joo. &不須駐口
    \end{tabularx}
\end{center}


\subsection{勉勵的祈使V-kini, uthai...okini}
\noindent 勉勵使令是-kini,重字上ci甯可說。就便如何遇整字,uthai下okini托。
\begin{center}
    \begin{tabularx}{\textwidth}{XX}
        bithe h\={u}lara niyalma g\={u}nin were\v{s}ekini. &讀書之人留心\\
        alban gara niyalma wesihun ibenekini. &當差之人向上\\
        butui erdemu be ambula iktambukini. & 陰隲多多的積\\
        mujilen -i gaman be saikan ilibukini. & 心術好好的培\\
        beyebe muribuci muribukini, niyalma be kokiraci ojorak\={u}. & 甯可屈己,不可損人。\\
        aniya hosime bithe h\={u}larak\={u} oci okini. & 甯可終歲不讀書\\
        uthai mujin acabukini, inu beyebe elehe araci ojorak\={u}. & 即便得志,也不可自足。\\
        uthai mentuhun albatu okini, inu bithe h\={u}labuci acambi. & 就便愚陋,亦當使令讀書。
    \end{tabularx}
\end{center}
%67

\subsection{“任憑、總然”ai hacin -i...V-kini, eitereme...V-kini}
\noindent ai hacin -i與eitereme,總然任憑語句說。句下應用-kini字,或用seme亦使得。
\begin{center}
    \begin{tabularx}{\textwidth}{XX}
        ai hacin -i bengsen bikini, inu dabure be ak\={u}. & 總然有本事,也算不了什麼。\\
        ai hacin -i sure genggiyen seme, inu banitai same muterak\={u}. & 任憑怎樣聰明,也不能生而知之。\\
        eitereme yangseleme miyamikini, inu ini ehe be daldame muterak\={u}. & 總然修飾,亦莫掩其惡。\\
        eitereme \v{s}orgime bo\v{s}oho seme, an -i ler seme elehun -i bimbi. & 任其催促,仍就逍遙自在。
    \end{tabularx}
\end{center}

\subsection{命令式V-cina, V-kini, ocina/okini}
\noindent -cina, -kini講是呢,口氣輭硬要明白。-cina本是自然句,-kini使令口氣多。ocina與okini,亦同此意一樣說。
\begin{center}
    \begin{tabularx}{\textwidth}{XX}
        sula \v{s}olo de beri tatacina. gabtame wajifi niyamniyara be urebukini. & 閒空兒拉拉弓是呢,射了步箭練練馬箭是呢。\\
        ume dabali mamgiyame fayara, majige hibcan ocina.& 不要過於耗費了,稍省儉罷。\\
        bayaka seme ai baita, ainame yadah\={u}n okini. & 富了怎樣,甯可窮罷。
    \end{tabularx}
\end{center}

\subsection{“聽見、聞得”donjici...se-}
\noindent 聽見、聞得是donjici,下用seme、sembi、sehe。
\begin{center}
    \begin{tabularx}{\textwidth}{XX}
        donjici, ai te ubaliyambure be tacimbi seme. &聽見你如今學繙譯呢。\\
        donjici, boode banjire doro, sain be yaburengge umesi sebjen sembi. & 聞得居家之道,為善最樂。\\
        donjici, gocish\={u}n oci nonggibumbi, jalu oci ekiyembi sehe. & 蓋聞謙受益,滿招損。
    \end{tabularx}
\end{center}

\subsection{“好不...!/How...!So...!”absi}
\noindent 好不、怎之是absi。又作何之、何往說。
\begin{center}
    \begin{tabularx}{\textwidth}{XX}
        absi hojo & 好不暢快\\
        absi koro & 好不傷感\\
        absi ferguwecuke & 好奇怪\\
        absi encu & 好別致\\
        absi oki sembini & 要怎之呢\\
        absi ojoro be sark\={u} & 不知怎之好\\
        absi geneki sembi & 將何之\\
        absi genembi & 何往
    \end{tabularx}
\end{center}

\subsubsection{“不拘、如何、無論”absi ocibe}
\noindent 不拘、如何、無論怎樣,無入焉往是absi ocibe.
\begin{center}
    \begin{tabularx}{\textwidth}{XX}
        absi ocibe ojorak\={u}ngge ak\={u}. & 不拘怎樣沒有使不得的\\
        absi ocibe inu gemu emu adali. & 無論怎樣也都是一般\\
        absi ocibe beye elehun ak\={u}ngge ak\={u}. & 無入而不自得焉\\
        absi ocibe acanarak\={u}ngge bio. & 焉往而有不宜者乎
    \end{tabularx}
\end{center}

\subsection{“誠以、蓋以、原以”cohome}
\noindent 誠以、蓋以、原以、凡以,併所以、無非乃cohome。g\={u}nin、jalin、turgun,整字長音俱可托。上文頓住另叫起,則用ere cohome。
\begin{center}
    \begin{tabularx}{\textwidth}{XX}
        julgei niyalma banjire be waliyafi jurgan be gaihangge, cohome jab\v{s}an de guwere anggala, jurgan be tuwakiyarangge wesihun de isirak\={u} sere turgun. & 古人舍生以取義者,誠以幸免,不如守義之為貴也。\\
        yaya bithei urse be ujire kesi, bithei urse be tacihiyara doro be, yooni umesi ak\={u}mbuhangge, cohome bithei niyalma be duin irgen -i irgen -i uju sere turgun. & 凡所以養士之恩,教士之法,無不至盡,蓋以士為四民之首也。\\
        tacik\={u} tacihiyak\={u} be ilibuhangge, cohome niyalmai erdemu be h\={u}wa\v{s}abuki, geren -i tacin be jiramilabuki jalin. & 學校之設,原以成人才而厚風俗。\\
        tere tacibuk\={u} hafan -i oronde, yooni tukiyese silgasi sebe baitalabuhangge, cohome saisa be yendebume erdemu be h\={u}wa\v{s}abuki, irgen be wembume, tacin be \v{s}anggabuki sehengge. & 其廣文一官,悉以孝廉明經取用,凡以興賢育才,化民成俗也。\\
        gurun boo fafun ilibuhangge, cohome ehengge be isebume, sain ak\={u}ngge be targabure jalin. & 朝廷之立法,所以警不善而懲無良。\\
        enduringge niyalma jalan be ulhibume irgen be tacibuhangge, cohome abkai fejergingge be gemu tob de forobukini sere g\={u}nin. & 聖人覺世牖民,無非使天下胥歸於正也。\\
        dorolon -i nomun de, kumdungge be jafara de jalungge be jafara adali, untuhun bade dosire de niyalma bisire gese sehengge, cohome niyalma de ginggun be da obure tacin be tacibuhangge kai. & 《禮》云,執虛如執盈,進虛若有人,乃教人主敬之學也。\\
        tere dulimba be teng seme jafa sehengge, cohome yoo han -i \v{s}\={u}n de afabuha mujilen -i doro kai. & 允執厥中者,是堯授舜之心法也。\\
        dasan -i nomun de, erdemu sehengge toktoho sefu ak\={u}, sain be da ararangge be sefu obuhabi sehengge, cohome niyalma be sain be sonjofi dahakini serengge kai. & 《書》曰,德無常師,主善為師,乃是教人擇善而從也。\\
        dung dz -i henduhengge, tumen irgen -i aisi be daharangge, uthai mukei wasih\={u}n ici eyere adali, tacihiyan wen -i tosome serem\v{s}erak\={u} oci, nakame muterak\={u} ombi sehebi. ere cohome irgen be dasarangge, bira be dasara ci dabanambi. dorolon jurgan -i tacihiyame wemburak\={u} oci, terei buyen be ilibuci ojorak\={u} sere turgun. & 董子曰,萬民之從利也,猶水之走下,不以教化隄防之,不能已也。誠以治民甚於治川,不以禮義教化之,其欲不可遏也。
    \end{tabularx}
\end{center}
%71

\subsection{“有”bi, “無”ak\={u}}
\noindent 有無當繙bi、ak\={u},不拘長短上有de。
\begin{center}
    \begin{tabularx}{\textwidth}{XX}
        yabun be toktobure de kemun bi, saisa be baitalara de memeren ak\={u}. & 制行有物,立賢無方。\\
        baita de toktoho giyan bi, irgen de entehen mujilen ak\={u}. & 事有定理,民無恆心。
    \end{tabularx}
\end{center}

\subsection{“在/不在”...de bi/ak\={u}}
\noindent de bi又作在字用,de ak\={u}作不在說。
\begin{center}
    \begin{tabularx}{\textwidth}{XX}
        dasan be yabubure oyonggongge, jemden be geterembure de bi. aisi be yendebure de ak\={u}. & 為政之要,在除弊,不在興利。
    \end{tabularx}
\end{center}

\subsection{“難/易”[V-ci, de] mangga, ja}
\noindent 難易繙為mangga、ja,上用長音或-ci/de
\begin{center}
    \begin{tabularx}{\textwidth}{XX}
        beyeingge be etere de mangga, amurangga be daharangge ja. & 克己難,從好易\\
        ejen ojorongge mangga, amban ojorongge ja ak\={u}. & 為君難,為臣不易\\
        yaya baita tuwaci ja bicibe, yabuci mangga. & 凡事看之易而作之難\\
        alin tafafi tasha be jafara de ja, angga juwame niyalma de alarangge mangga. & 上山擒虎易,開口吿人難。
    \end{tabularx}
\end{center}

%72
\subsubsection{“容易”ja -i}
\noindent 若是單言容易字,上用ja -i也使得。
\begin{center}
    \begin{tabularx}{\textwidth}{XX}
        dursun bisire jaka be, ja -i weilembi. kooli bisire baita be ja -i icihiyambi.& 有式之物容易作,有例之事容易辦。\\
        ja -i yabure jug\={u}n be yaburak\={u}, ja -i bodoro boje be bodorak\={u}. & 好走的道路不走,好算的賬目不算。
    \end{tabularx}
\end{center}

\subsection{“善於”V-rA mangga (注意:V-rA de mangga是做起來難)}
\noindent mangga又作善、肯用,字上須接-ra -re -ro。
\begin{center}
    \begin{tabularx}{\textwidth}{XX}
        niyalma takara mangga. & 善識人\\
        boo banjire mangga. & 善居家\\
        tubi\v{s}eme bodoro mangga. &善揣摩\\
        bithe h\={u}lara mangga. & 肯讀書\\
        g\={u}nin were\v{s}ere mangga. & 肯留神\\
        baita onggoro mangga. & 肯忘事
    \end{tabularx}
\end{center}

\subsection{“凡、各”yaya, yaya...seme}
\noindent 凡字則當繙yaya,不拘無論下用seme。
\begin{center}
    \begin{tabularx}{\textwidth}{XX}
        yaya niyalma tacin de, giyan -i g\={u}nin sith\={u}ci acambi. & 凡人為學,理當用心。\\
        yaya we seme, bahanarak\={u}ci ojorak\={u}. ele sark\={u}ci ojorak\={u}. & 不拘是誰,不可不會,尤不可不知。\\
        yaya ai baita seme, biyan be dahame yabuci sain. & 無論何事,可得按理而行。
    \end{tabularx}
\end{center}

\subsection{“是、亦、也”inu}
\noindent inu本繙是、亦、也,上用長音整字托。句上平說為也亦,用在字下是也說。
\begin{center}
    \begin{tabularx}{\textwidth}{XX}
        bihe bihei, inu ehe ome dubihe. & 一來而去,也慣壞了。\\
        entekengge inu sesheri niyalma de duibuleci ojorak\={u}. & 似此者,亦非俗人可比。\\
        mujilen serengge, emu beyei da inu. & 心是一身之主。\\
        unenggi oci hafurak\={u}ngge ak\={u} sehengge inu. & 所謂誠無不格也。
    \end{tabularx}
\end{center}

\subsection{“該、當、宜、應”giyan, giyan -i, V-rA giyan, V-ci giyan}
\noindent 該當宜應皆繙giyan,或上或下皆使得。用在句首下連-i,在下上接-ra -re -ro,往往也托長音等,ci字間或亦使得。
\begin{center}
    \begin{tabularx}{\textwidth}{XX}
        giyan -i ehe be halafi, sain de foroci acambi. & 理當改惡向善\\
        neileme yarh\={u}dara giyan & 應開導\\
        teodejeme forgo\v{s}oro giyan & 該調換\\
        cibtui g\={u}nire giyan & 當三思\\
        da be waliyafi, dube be kicerengge giyan & 宜乎捨本而逐末也\\
        beye dorolome geneci giyan & 理應親身往拜\\
        jasigan jasifi hebdeci giyan & 理當寄信相商
    \end{tabularx}
\end{center}

\subsection{“許久、良久”kejine}
\noindent kejine上許久、良久,在下則又講多多。
\begin{center}
    \begin{tabularx}{\textwidth}{XX}
        kejine ofi teni mariha. & 許久方歸\\
        kejine oho manggi teni aituha &許久方蘇\\
        emu hergen seme takarak\={u}ngge kejine bi. & 多有一丁不識者\\
        ere tere seme ilgarak\={u}ngge kejine bi. & 不分彼此者多多矣
    \end{tabularx}
\end{center}

\subsection{“越發、更加”ele, V-rA/hA ele}
\noindent ele是越發、益加講,又作更、彌、尤、愈說。若在-ra -re -ha -he下,又作諸、凡、所有說。
\begin{center}
    \begin{tabularx}{\textwidth}{XX}
        ele baita waka oho. & 越發不成事體\\
        ele kiceme fafur\v{s}ambi. & 益加勤奮\\
        ele narh\={u}\v{s}ame olho\v{s}oci acambi. & 更當詳慎\\
        ele kimcici ojorak\={u}. & 更不可考\\
        harga\v{s}aci ele den. & 仰之彌高\\
        \v{s}orgici ele akdun. & 鑽之彌堅\\
        tere arbun ele saikan. & 其形尤美\\
        tere tuwabun ele sain. & 其况尤佳\\
        kicerengge ele nemeci, dosinarangge ele \v{s}umin ombi. & 功愈加,而進愈深\\
        ton arara ele urse, yerte\v{s}erak\={u} mujanggao. & 凡諸充數者,甯不慚愧\\
        yamun de bisire ele hafasa, ainame ainame gamaci ombio. & 所有在署之員,豈可因循\\
        ichihiyaha ele baita be, kooli hacin de dosimbu. & 凡辦過案件,著載入則例\\
        selgiyeme tucibuhe ele hesei bithe be, folofi suwayan hoo\v{s}an de \v{s}uwasela. & 所有頒發詔㫖,著刊刻謄黃。
    \end{tabularx}
\end{center}

\subsection{“所有的XX”-le}
\noindent 暗含所有繙-le字,文意猶如用ele。上靣多連-ha -he字,下加長音亦有則。
\begin{center}
    \begin{tabularx}{\textwidth}{XX}
        ucarahale niyalma, niyalma h\={u}ncihin jaka oci, uthai fe gucu ombi. & 所遇之人,非親即故。\\
        dulekele ba, necin sain haksan olhocukangge adali ak\={u}. & 所過地方,平夷險阻不同。\\
        eiten bade tehelengge, han -i aha wakangge ak\={u}. & 率土之濱,莫非王臣。
    \end{tabularx}
\end{center}

\subsection{“強如、爭似”...ci ai dalji, “何干、何與”...de ai dalji}
\noindent 強如、爭似、何干、何與,ai dalji字句多。上接ci字強如、爭似,何干、何與上連de。
\begin{center}
    \begin{tabularx}{\textwidth}{XX}
        hiyang daburangge, juktehe sara\v{s}ara ci ai dalji. & 燒香強如逛廟\\
        erdemu iktamburengge aisin werire ci ai dalji. & 積德爭似遺金\\
        bi dacibe darak\={u} ocibe, sinde ai dalji. & 我管不管與你何干?\\
        niyalmai yendere jocirengge, jaka de ai dalji. & 人之興替,與物何與?
    \end{tabularx}
\end{center}

\subsection{“連連”nurh\={u}me}
\noindent 連累、屢見是nurh\={u}me,上連整字看文波。
\begin{center}
    \begin{tabularx}{\textwidth}{XX}
        aniya nurh\={u}me elgiyen -i bargiyaha. & 連年豐收\\
        mudan nurh\={u}me dabame oloho. & 累次跋涉\\
        aniya nurh\={u}me haji yuyun. & 屢次荒歉\\
        mudan nurh\={u}me kesi isibuha. & 疊次施恩
    \end{tabularx}
\end{center}

%76
\subsection{“連着、帶著”nisihai}
\noindent 連着帶著是nisihai,上接整字是規模。
\begin{center}
    \begin{tabularx}{\textwidth}{XX}
        jasigan be jise nisihai suwaliyame gaju. & 信連着底子一併取來。\\
        bithe be dobton nisihai gemu h\={u}balaki. & 書帶套都裱一裱。\\
        da jergi nisihai tusan ci nakabu. & 原品休致。
    \end{tabularx}
\end{center}

\subsection{“光是、竟是”noho}
\noindent 光是、竟是用noho,整字與-i上連着。
\begin{center}
    \begin{tabularx}{\textwidth}{XX}
        yasai juleri emu girin -i ba, gemu jahari noho. & 眼前一帶,光是石子。\\
        duin ici jug\={u}n ak\={u}, muke noho. & 四下無路,竟是水。
    \end{tabularx}
\end{center}

\subsection{“儘其所能”muterei teile}
\noindent muterei teile儘其力量,上用-i ni字接者。
\begin{center}
    \begin{tabularx}{\textwidth}{XX}
        h\={u}sun -i muterei teile yabumbi. & 儘其力而為之\\
        encehen -i muterei teile kicembi. & 儘其量而圖之\\
        mini muterei teile kiceki. & 儘我而謀\\
        ini muterei teile yabuki. & 儘他而作
    \end{tabularx}
\end{center}

\subsection{“儘數”ebsihe}
\noindent 儘數則用ebsihe,上連-i字與-ra -re。如今下連-i字用,ebsihei字句多。
\begin{center}
    \begin{tabularx}{\textwidth}{XX}
        ton -i ebsihe afabume tucibuhe. & 儘數交出\\
        bisire ebsihe gemu gamaha. & 儘其所有都拿去了\\
        beyei bisire ebsihe ujibumb. & 以養餘年
    \end{tabularx}
\end{center}

\subsection{“既”manggi, nak\={u}}
\noindent manggi, nak\={u}講既字,上用半截字連着。彼而又此使manggi,然而未果nak\={u}合。
\begin{center}
    \begin{tabularx}{\textwidth}{XX}
        duka uce be ambarame eldembu manggi, geli amaga enen de elgiyen -i tutabumbi. & 既光大門閭,又垂裕後昆。\\
        gebu eje manggi, geli ilgame gisurebuhe. & 既經記名,復蒙議敘。\\
        getukeleme wesimbufi yabubu nak\={u}, geli jurgan ci gisurefi bederebumbi. & 既奏明允准,又准部議駁。\\
        nonggime dosimbu nak\={u}, dahanduhai sib\v{s}alame meitehe. & 既經添入,旋即裁汰。
    \end{tabularx}
\end{center}

\subsection{“次、囘、番”mari, -nggeri, mudan}
\noindent mari, -nggeri, mudan是遭次,整單破連用法多。
\begin{center}
    \begin{tabularx}{\textwidth}{XX}
        acame udu mari ohak\={u}, amala uthai tulergi goloi hafan de wesinehe. & 見不幾次,後來就陞出外任官去了。\\
        ududu mari jifi, \v{s}uwe bahafi dere acahak\={u}. & 來了好幾次,竟未得見靣。\\
        amasi julesi ududu juwanggeri yabufi, amala teni teyehe. & 來往走了數十遭,然後纔歇下了。 \\
        ilan mudan cooha tucifi, ninggunggeri afaha. & 出了三次兵,打了六次仗。\\
        mudan mudan -i beyebe tuwabuha de, gemu cohohongge be sindahabi. & 歷次引見,都放了擬正的了。\\
        mudan nurh\={u}me alban kemu yabuhangge, yala umesi suilahabi. & 連次當差,實在狠累。
    \end{tabularx}
\end{center}

\subsection{“每”+數詞-ta, -te, “每”+年月日時-dari, “每”+名詞tome}
\noindent 每繙-ta, -te, dari, tome,數目多寡用-ta, -te,年月日時用-dari,地方人物用tome。
\begin{center}
    \begin{tabularx}{\textwidth}{XX}
        ere tubihe be niyalma tome inenggidari ilata fali jefu. & 這果子每人每日喫三個。\\
        tere suje be boo tome biyadari duite ju\v{s}uru bene. & 那縀子每家每月送四尺。
    \end{tabularx}
\end{center}

\subsection{“每、逐、逐一”aname}
\noindent aname變文借用字,亦作每字意思說。
\begin{center}
    \begin{tabularx}{\textwidth}{XX}
        bithe tuwara kooli, hergen tome gisun aname, gemu fuha\v{s}ame kimcime sibkime baicaci acambi. & 讀書之法,逐字逐句,皆當玩索而研究之。\\
        haji habcihiyan -i tacin, ga\v{s}an aname hoton tome iletulefi, h\={u}waliyasun sain -i sukdun gubci mederi -i dorgi tulergi de hafunambi. & 親睦之風,成於一鄉一邑,雍和之氣,達於薄海內外。
    \end{tabularx}
\end{center}

\subsection{“即至、以及”...ci aname}
\noindent 即至以及轉入語,句中用ci aname。
\begin{center}
    \begin{tabularx}{\textwidth}{XX}
        usisi faksi h\={u}da\v{s}ara tuwelere urse, gulu nomhon ojoro be ufararak\={u} sere anggala, selei hitha nerere urse ci aname, inu dorolon kumun irgebun bithe de yumpi wembufi, ini hatan foro cokto kangsanggi be dorgideri mayambuci ombi. & 農工商貿,不失為湻樸,即至韜鈴介冑之士,亦被服乎禮樂詩書,以潛消其剽悍桀驁。\\
        gemulehe hecen amba hoton de, elgiyen -i bargiyaha jalin uhei urgunjendumbi sere anggala, simeli ga\v{s}an mudan wai ci aname, angga de a\v{s}ume hefeli be bi\v{s}ume sebjelerak\={u}ngge ak\={u} kai. & 通都大邑,共慶豐稔,以及窮鄉僻壤,未有不含餔鼓腹而樂者也。
    \end{tabularx}
\end{center}

\subsection{形容詞指小-kan, -ken, -kon}
\noindent 整字微乎-kan -ken -kon。
\begin{center}
    \begin{tabularx}{\textwidth}{XX}
        h\={u}wa -i fu daldangga fu ci fangkalakan. & 牆院比影壁微乎矮些\\
        cin -i boo julergi boo ci deken. & 正房比南房微乎高些\\
        celehe jug\={u}n talu jug\={u}n ci oncokon. & 甬路比夾道微乎寬些
    \end{tabularx}
\end{center}

\subsection{動詞指小V-meliyen}
\noindent 破字微然-meliyen多。
\begin{center}
    \begin{tabularx}{\textwidth}{XX}
        gisureme goiha manggi, injemeliyen -i umai her har serak\={u}. & 說是了,微然笑一笑不作理會。\\
        ere helmen e\v{s}emeliyen ofi, asuru tob ak\={u}. & 這影兒微然斜一點兒,不大很直。\\
        doko jug\={u}n deri majige mudalimeliyen bicibe, inu asuru calabure ba ak\={u}. & 由小道微然繞一點兒,也差不多。
    \end{tabularx}
\end{center}

\subsection{“稍微”V-sh\={u}n, V-shun}
\noindent 又有-sh\={u}n, -shun字,亦作稍微意思說。
\begin{center}
    \begin{tabularx}{\textwidth}{XX}
        golmish\={u}n & 稍微長些\\
        sartash\={u}n & 稍微誤些\\
        enggeleshun & 稍前探些\\
        tukiyeshun & 微挺著些
    \end{tabularx}
\end{center}

\subsection{“往XX”-sih\={u}n, -sihun, -si, baru}
\noindent -sih\={u}n、-sihun與-si字,baru的神情往向說。
\begin{center}
    \begin{tabularx}{\textwidth}{XX}
        wasih\={u}n & 往西\\
        wesihun & 往東\\
        julesi & 向前\\
        amasi & 向後\\
        mini baru miosirilambi & 望着我冷笑\\
        ini baru menerembi & 望著他發獃
    \end{tabularx}
\end{center}

\subsection{極盡-tai, -tei, 堪可-cukA}
\noindent 極盡意思用-tai、-tei,堪可神情-cuka、-cuke。此等用法皆一體,上將半截字連着。
\begin{center}
    \begin{tabularx}{\textwidth}{XX}
        nure omire de watai amuran soktorak\={u} oci nakarak\={u}. & 極好飲酒,不醉不止。\\
        bucetei fa\v{s}\v{s}ame, funcehe h\={u}sun be hairarak\={u}. & 盡命奮勉,不惜餘力。\\
        tondo amban hiyoo\v{s}ungga jui -i jekdun jilihangga sai\v{s}acuka. & 忠臣孝子貞烈可嘉\\
        mergen niyalma fujurungga saisa -i yabun dursun buyecuke. & 淑人秀士風度可羨\\
        yadara duka suilash\={u}n boo -i yadah\={u}n hafirah\={u}n jobocuka. & 蓬門華戶窮迫堪虞\\
        h\={u}lha holo ehe urse -i doksin oshon gelecuke. & 賊盜匪徒暴虐可懼
    \end{tabularx}
\end{center}

\subsection{形容詞詞尾、名物化-ngga, -ngge, -nggo}
\noindent 整字連寫-ngga, -ngge, -nggo,乃作有字人字說。
\begin{center}
    \begin{tabularx}{\textwidth}{XX}
        usihangga & 有知識\\
        ulhicungga & 有悟性\\
        funiyagangga & 有度量\\
        serecungge & 有眼色\\
        ferkingge & 有見識\\
        bengsengge & 有本事\\
        doronggo & 有道理\\
        horonggo & 有威嚴\\
        bodohonggo s& 有謀畧\\
        fulingga & 厚福人\\
        jilangga & 慈善人\\
        fulungga & 安常人\\
        encehengge & 能幹人\\
        sebsingge & 和氣人\\
        yebkengge & 俊傑人\\
        kobtonggo & 敬謹人\\
        tomohonggo & 鎮定人\\
        \v{s}olonggo & 爽俐人
    \end{tabularx}
\end{center}

\subsection{敢gelhun ak\={u}/V-rak\={u}, 豈敢ai gelhun ak\={u}/V-rak\={u}}
\noindent gelhun ak\={u}是敢字,ai gelhun ak\={u}豈敢說,文中若遇不敢字,ak\={u}, -rak\={u}下靣托。
\begin{center}
    \begin{tabularx}{\textwidth}{XX}
        gelhun ak\={u} uttu gisureci tetendere, uthai gelhun ak\={u} songkoi yabumbi. & 既敢這們說,就敢照着行\\
        ai gelhun ak\={u} erdemu be juwedeme iladambini. & 不敢辜負背德\\
        gelhun ak\={u} mujilen be ak\={u}mburak\={u} h\={u}sun be wacihiyarak\={u} oci ojorak\={u}. & 不敢不盡心竭力\\
        gelhun ak\={u} tondo ak\={u}ngge ak\={u}. & 不敢不忠
    \end{tabularx}
\end{center}

\subsection{“不勝”alimbaharak\={u}}
\noindent 不勝、無任alimbaharak\={u}。
\begin{center}
    \begin{tabularx}{\textwidth}{XX}
        den be huk\v{s}eme jiramin be fehurengge, alimbaharak\={u} dolori hing seme huk\v{s}embi. amban be alimbaharak\={u} geleme olhombime urgunjeme selambi. & 載高履厚,不勝感激於衷,臣等無任悚惶祈忭。
    \end{tabularx}
\end{center}

\subsection{“不覺得”hercun ak\={u}}
\noindent hercun ak\={u}不覺得。
\begin{center}
    \begin{tabularx}{\textwidth}{XX}
        hercun ak\={u} dursuki ak\={u} de eyenefi, aitubuci ojorak\={u} ohobi. & 不覺得流於不肖而不可救。\\
        yasa habta\v{s}ara sidende, hercun ak\={u} geli emu aniya oho. & 展眼閒,不覺又是一年。
    \end{tabularx}
\end{center}

\subsection{“不圖、不想、不憶”g\={u}nihak\={u}}
\noindent 不憶不圖是g\={u}nihak\={u},文中乃字也繙得。
\begin{center}
    \begin{tabularx}{\textwidth}{XX}
        g\={u}nihak\={u} sain nomhon urse, enteke jobolon gashan de tu\v{s}aha. & 不意善良輩遭次荼毒。\\
        kumun -i uttu oho be g\={u}nihak\={u} bihe. & 不圖為樂之至於斯也。\\
        g\={u}nihak\={u} inenggi aname genehei, h\={u}sibume dalibuhangge umesi \v{s}umin ohobi. & 乃日復一日,痼弊已深。
    \end{tabularx}
\end{center}

\subsection{“不下、不亞、不讓”...ci eberi ak\={u}}
\noindent eberi ak\={u}上連ci,不下、不亞、不讓說。
\begin{center}
    \begin{tabularx}{\textwidth}{XX}
        geren dz tangg\={u} boo -i bithe, ududu minggan tumen hacin ci eberi ak\={u}. & 諸子百家之書,不下數千萬種。\\
        judz -i tacin mengdz ci eberi ak\={u}. & 朱子之學,不亞孟子。\\
        wesihun jalan -i dasan duleke julge ci eberi ak\={u}. & 熙朝之治,不讓往古。
    \end{tabularx}
\end{center}

\subsection{“不成、不行、不來、不了”banjinarak\={u}}
\noindent 不成不行不來不了,banjinarak\={u}上緊接之。
\begin{center}
    \begin{tabularx}{\textwidth}{XX}
        geneme banjinarak\={u}. & 去不成\\
        gisureme banjinarak\={u}. & 說不行\\
        icihiyame banjinarak\={u}. & 辦不來\\
        yabume banjinarak\={u}. & 行不了
    \end{tabularx}
\end{center}

\subsection{“不得已”umainaci ojorak\={u}}
\noindent umainaci ojorak\={u}不得已。
\begin{center}
    \begin{tabularx}{\textwidth}{XX}
        fafun kooli serengge, gurun boode umainaci ojorak\={u} ilibuhangge. & 法例者,朝廷不得已而設也。\\
        ere baita be umesi umainaci ojorak\={u} ofi yabuha dabala. & 此事出於萬不得已耳。
    \end{tabularx}
\end{center}

\subsection{“不由得”esi seci ojorak\={u}}
\noindent esi seci ojorak\={u}不由得。
\begin{center}
    \begin{tabularx}{\textwidth}{XX}
        uttu sehe be donjire jakade, esi seci ojorak\={u} dolori se selahabi. & 聽見如此說,不由得心裏暢快。\\
        fili fiktu ak\={u} balai dai\v{s}ahangge, tuwaci esi seci ojorak\={u} jili banjimbi. & 無故的混鬧,瞧之不由得生氣。
    \end{tabularx}
\end{center}

\subsection{“一定、必定”urunak\={u}}
\noindent 務須、必定是urunak\={u}。
\begin{center}
    \begin{tabularx}{\textwidth}{XX}
        urunak\={u} yargiyan fulehun be neigenjeme bahabu, \v{s}udesi yamun -i ursei cihai jemden be deribubuci ojorak\={u}. &務須使實惠均沾,毋得任胥吏舞弊。\\
        sain be yabuci urunak\={u} sain karulan bahambi. & 行善必定得好。
    \end{tabularx}
\end{center}

\subsection{“偏是、一定”urui}
\noindent 偏是一定urui合。
\begin{center}
    \begin{tabularx}{\textwidth}{XX}
        urui ferguwecuke be kiceme fulu be buyere mangga. & 偏是肯務奇好勝\\
        urui aldungga be gaime ice be temgetuleki sembi. &一定要領異標新\\
    \end{tabularx}
\end{center}

\subsection{“能幾何”giyanak\={u}}
\noindent 能有多少是giyanak\={u}。
\begin{center}
    \begin{tabularx}{\textwidth}{XX}
        sain erdemungge niyalma giyanak\={u} udu? & 善德人能有多少\\
        urgun sebjen -i baita giyanak\={u} udu? &喜歡事能有多少\\
        giyanak\={u} udu salimbi? &能值幾何\\
        giyanak\={u} udu funcembi. &能賸多少
    \end{tabularx}
\end{center}

\subsection{副詞“博大、廣多”ambula}
\noindent ambula是博大廣多。
\begin{center}
    \begin{tabularx}{\textwidth}{XX}
        ambula donjifi hacihiyame ejembi. & 博聞強記\\
        ambula tusangga babi. & 大有裨益\\
        tafulara jug\={u}n be ambula neihe. & 廣開言路\\
        ulin nadan be ambula isabuha. & 多聚資財
    \end{tabularx}
\end{center}

\subsection{副詞“特、專”cohotoi}
\noindent 特特專專cohotoi。
\begin{center}
    \begin{tabularx}{\textwidth}{XX}
        cohotoi dorolome gah\={u}\v{s}ambi. & 特特拜懇\\
        cohotoi elhe be baime jihe. & 特特來請安\\
        cohotoi ereme g\={u}nimbi. & 專專指望\\
        cohotoi kicere tacini & 專門之業
    \end{tabularx}
\end{center}

\subsection{“不干、無涉、無關”daljak\={u}}
\noindent daljak\={u} 不干、無與、無涉。\\
\begin{center}
    \begin{tabularx}{\textwidth}{XX}
        beyede daljak\={u} -i baita be gala dafi ainambi. & 不干己事,何必著手。\\
        suwede daljak\={u} seci tetendere, uthai dabuci ojoro ba ak\={u}. & 既是與爾等無與,就無足輕重。\\
        minde daljak\={u} -i adali, ainaha seme dolo tata\v{s}arak\={u}. & 似乎與我無涉,絕不介意。
    \end{tabularx}
\end{center}

\subsection{稍微heni, 稍加majige}
\subsubsection{稍微heni}
\noindent 微乎些須是heni,幾希用在句尾托。\\
\begin{center}
    \begin{tabularx}{\textwidth}{XX}
        heni amtan bi. heni baita ak\={u}. & 微乎有點味,無有些須事。\\
        niyalma gasha gurgu ci encungga heni. & 人之異於禽獸者幾希。
    \end{tabularx}
\end{center}

\subsubsection{稍加majige}
\noindent 少少畧畧微微的,稍加一些是majige。
\begin{center}
    \begin{tabularx}{\textwidth}{XX}
        majige muke sinda. & 少少着點水\\
        majige cendeme tuwa. & 畧畧的試一試\\
        majige yebe oho. & 微微的好了些\\
        majige meiteme halaha. & 稍加刪改\\
        majige fulu ba ak\={u}. & 無有一些長處\\
        majige tesurak\={u} babi. & 稍有不足
    \end{tabularx}
\end{center}

\subsubsection{一點點heni majige}
\noindent heni majige連着用,一毫一點意思說。
\begin{center}
    \begin{tabularx}{\textwidth}{XX}
        heni majige jalingga koimali -i mujilen deribuci ojorak\={u}. & 不可起一毫奸詐之心\\
        heni majige untuhun holo -i g\={u}nin tebuci ojorak\={u}. & 不可存一點虛偽之見\\
        heni majige fulu gaiha ba ak\={u}. & 未曾一毫多取\\
        heni majige cisu jemden ak\={u}. & 沒有一點私弊
    \end{tabularx}
\end{center}

\subsection{仍然、照常an -i}
\noindent 仍然照常使an -i。
\begin{center}
    \begin{tabularx}{\textwidth}{XX}
        an -i fe songkoi yabumbi. & 仍然照舊而行\\
        an -i baita icihiyambi. & 照常辦事
    \end{tabularx}
\end{center}

\subsection{依舊、仍舊da an -i}
\noindent da an -i依舊、仍舊說。
\begin{center}
    \begin{tabularx}{\textwidth}{XX}
        da an -i dah\={u}me muheliyen oho. & 依舊復圓\\
        da an -i bici antaka. & 仍舊貫如之何\\
        da an -i ojoro unde. & 仍未復元
    \end{tabularx}
\end{center}

\subsection{漸漸、漸次ulhiyen}
\noindent ulhiyen -i漸次、馴日字,重言蒸蒸、駸駸多。
\begin{center}
    \begin{tabularx}{\textwidth}{XX}
        geren goloi coohai baita, ulhiyen -i bolgo oho. & 各省軍務漸次就清\\
        badarambume gisureme, ulhiyen -i hing seme gungnecuke ofi abkai fejergi necin oho wesihun de isibuha. & 推而言之,以馴致乎篤恭而天下平之盛\\
        erdemu -i wembuci, niyalma ulhiyen -i sain de ibedembime sark\={u} ombi. & 化之以德,民日趨善而不知\\
        ulhiyen ulhiyen -i dasabufi, ehe de isinahak\={u}bi.& 蒸蒸乂,不格姦。\\
        ulhiyen ulhiyen -i erun baitalarak\={u} -i dasan de haminahabi. & 駸駸乎幾於刑措之治矣
    \end{tabularx}
\end{center}

\subsection{正在、正然jing}
\noindent 正然、正在繙jing字。
\begin{center}
    \begin{tabularx}{\textwidth}{XX}
        jing amasi jidere de, gaitai jug\={u}n -i andala teng seme ilinjaha. & 正然往回裏來,忽然半路站住了。\\
        jing bahakini seme ereme bisire de, yargiyan -i nash\={u}labuha be we g\={u}niha. & 正在指望這得,誰想真凑巧。
    \end{tabularx}
\end{center}

\subsection{正經、真實jingkini}
\noindent jingkini正經真實合
\begin{center}
    \begin{tabularx}{\textwidth}{XX}
        jingkini niyalmai yabun hiyoo\v{s}un deocin tondo akdun ci tucinarak\={u}. & 正經人品,不外孝弟忠信。\\
        jingkini tacin fonjin, dorolon kumun irgebun bithe de baktakabi. & 真實學問乃在禮樂詩書。\\
        jingkini ulan & 真傳\\
        jingkini obume sindambi. & 實授
    \end{tabularx}
\end{center}

\subsection{預先doigonde, doigom\v{s}ome}
\noindent 豫先繙譯有二者,doigonde與doigom\v{s}ome。doigom\v{s}ome是着力字,自然口氣doigonde。
\begin{center}
    \begin{tabularx}{\textwidth}{XX}
        doigonde same muterenge, endurin waka oci ai. & 豫先能知道,除非是神仙。\\
        doigom\v{s}ome belheme jabduci, ai ai baita gemu calabure de isinarak\={u} ombi. & 豫先豫備妥當,什麼事不致都錯誤。
    \end{tabularx}
\end{center}

\subsection{互相、彼此ishunde}
\noindent ishunde是互相、彼此
\begin{center}
    \begin{tabularx}{\textwidth}{XX}
        ishunde siltame anadambi. & 互相推諉\\
        ishunde aisilame wehiyembi. & 彼此扶持\\
        ishunde guculembi. & 相交\\
        ishunde g\={u}limbi. & 相契
    \end{tabularx}
\end{center}

\subsection{順便、乘便ildun de}
\noindent 順便、乘便是ildun de。
\begin{center}
    \begin{tabularx}{\textwidth}{XX}
        hoton tucire ildun de, bigan -i tuwabun be tuwanambi. & 出城順便,看看野景。\\
        alban kame yabure ildun de, niyalma h\={u}ncihin -i boode dariki. & 乘當差之便,往親戚家走走。
    \end{tabularx}
\end{center}

\subsection{自然ini cisui}
\noindent 自然本是ini cisui。
\begin{center}
    \begin{tabularx}{\textwidth}{XX}
        kicen isinaha manggi, ini cisui \v{s}anggambi. & 工夫到了自然成\\
        bihe bihei ini cisui tusa acabun bi. & 久而久之,自有效驗。
    \end{tabularx}
\end{center}

\subsection{反倒elemangga, nememe}
\noindent 反到elemangga與nememe。
\begin{center}
    \begin{tabularx}{\textwidth}{XX}
        si endebuhe bime, elemangga g\={u}wa niyalma be laidambi. & 你錯了,反訛別人。\\
        ineku arga ak\={u}, nememe mini baru ten gaimbi. & 本就無法,反到望我要準。
    \end{tabularx}
\end{center}

\subsection{即便uthai}
\noindent 即便就是uthai。
\begin{center}
    \begin{tabularx}{\textwidth}{XX}
        uthai uttu gisureci inu ojorak\={u} sere ba ak\={u}. & 即如此言之,亦未嘗不可。\\
        yaki seci, uthai yocina. ak\={u}ci, baji oho manggi, uthai sitaha. & 要走就走罷,不然,少遲一會,就誤了。\\
        ehe gisun tuciburak\={u}ngge, uthai sain niyalma inu. & 不說不好話,便是好人。
    \end{tabularx}
\end{center}

%89
\subsection{“總之,要之”eiterecibe}
\noindent 總之,要之eiterecibe.
\begin{center}
    \begin{tabularx}{\textwidth}{XX}
        eiterecibe, cib seme a\v{s}\v{s}arak\={u}ngge, dursun inu acinggayabuha ici hafunarangge, baitalan inu. & 總之,寂然不動者,體也,感而遂通者,用也。\\
        eiterecibe, enduringge niyalma doro unenggi ci tucinarak\={u}. & 要之,聖人之道,不外乎誠。
    \end{tabularx}
\end{center}

\subsection{“一一、逐一”emke emken -i}
\noindent emke emken -i一一、逐一。
\begin{center}
    \begin{tabularx}{\textwidth}{XX}
        tu\v{s}an tu\v{s}an -i yabuha ba be, emke emken -i getukeleme faidame araha. & 歷任履歷,一一開列明白。\\
        emke emken -i kimcime baicafi, geren hacin -i ton yooni acanaha.& 逐一考查,各款數目悉屬相符。
    \end{tabularx}
\end{center}

\subsection{“直、索”\v{s}uwe}
\noindent 直索豁貫竟繙\v{s}uwe
\begin{center}
    \begin{tabularx}{\textwidth}{XX}
        \v{s}uwe gemun hecen be baime jihe. & 直奔京師\\
        \v{s}uwe sain be tacirak\={u} & 索不好學\\
        \v{s}uwe neibufi g\={u}nin bahabuha. & 豁然而悟\\
        gaitai \v{s}uwe hafuka. & 忽爾貫通\\
        \v{s}uwe genefi amasi marirak\={u}. & 竟去而不返
    \end{tabularx}
\end{center}

\subsection{“差很遠,逈別、逈異、懸殊、懸隔”cingkai}
\noindent cingkai本講差很遠,逈別、逈異、懸殊、懸隔。
\begin{center}
    \begin{tabularx}{\textwidth}{XX}
        tacihangge inci cingkai calabuha. & 學的比他差遠了\\
        tehengge inci cingkai goro. & 住的比你遠的很\\
        erin -i arbun cingkai encu. & 時勢逈別\\
        an -i jergi niyalma ci cingkai encu. & 逈異尋常\\
        tob miosihon cingkai encu. & 邪正懸殊\\
        abka na -i gese cingkai giyalabuha. & 天地懸隔
    \end{tabularx}
\end{center}

\subsection{各meni meni, teisu teisu, beri beri, meimeni}
\noindent meni meni、teisu teisu、beri beri,與meimeni皆作各自說。meimeni、meni meni衆齊用,teisu teisu、beri beri各人合。
\begin{center}
    \begin{tabularx}{\textwidth}{XX}
        baita de afaha urse meimeni baita be kadalambi. & 執事者各司其事。\\
        meni meni teisu be ak\={u}mbumbi. & 各盡乃職。\\
        meni meni baita be tuwakiyambi. & 各世其業。\\
        cooha irgen \v{s}uwe teisu teisu huwakiyandume yendenume juse deote -i giyan be ak\={u}mbu. & 兵民等其感發,興起各盡子弟之職。\\
        funcehe h\={u}lha beri beri buruleme ukaha. & 餘匪各自逃竄
    \end{tabularx}
\end{center}

\subsection{副動詞“直至”V-tala, V-tele, V-tolo}
\noindent 直至永遠遲久意,-tala、-tele與-tolo。
\begin{center}
    \begin{tabularx}{\textwidth}{XX}
        sulaha tacin ulaha kooli, goroo goidatala tutabuha.& 遺俗流風,久遠貽留。\\
        hing sere tondo, teng sere mujin, tetele hono iletulehe. & 精忠鋭志,至今猶著。\\
        colgoroko gungge ferguwecuke lingge, ududu jalan otolo ulaha. & 奇勳偉烈,傳及數世。
    \end{tabularx}
\end{center}

\subsection{ombi用法}
\subsubsection{ome, ofi, ojoro}
\noindent 整字必要改破字,ome、ofi、ojoro。
\begin{center}
    \begin{tabularx}{\textwidth}{XX}
        unenggi ama eme de hiyoo\v{s}un, gucu gargan de akdun ome mutere ohode, ini cisui ga\v{s}an falin ulhiyen -i h\={u}waliyasun, falin adaki ulhiyen -i h\={u}wangga ofi, yabun tuwakiyan gulu yongkiyan, g\={u}nin mujilen gosin jiramin ojoro be dahame, erdemu \v{s}anggaha ambasa saisa serengge, enteke niyalma wakao? & 誠能孝於父母,信乎朋友,自然鄉黨日和,鄰里日睦,將見品行醇全,心地仁厚,而成德之君子,其非斯人與?
    \end{tabularx}
\end{center}

\subsubsection{“已、可以、作為”oho, ombi, obure}
\noindent 已了、可以、作為字,oho、ombi、obure。ombi可以上接ci,其餘整字方接得。
\begin{center}
    \begin{tabularx}{\textwidth}{XX}
        jug\={u}n be dasatahangge, emgeri necin oho. & 道墊的已平了。\\
        nimeku be oktosilahangge emgeri yebe oho. & 病治的已好了。\\
        ere niyalma de heo seme guculeci ombi. & 此人頗可交。\\
        tere baita be lak seme icihiyaci ombi. & 其事可以辦。\\
        julgei niyalma be durun obure dabala, te -i niyalma be ume kooli obure. & 以古人作則,別以今人為法。
    \end{tabularx}
\end{center}

\subsubsection{“因為、可以”...oci ofi, ...ome ofi}
\noindent oci、ome連ofi,乃是因為可以說。
\begin{center}
    \begin{tabularx}{\textwidth}{XX}
        si onggorak\={u} oci ome ofi, teni sinde afabufi simbe ejebuhebi. & 因為你可以不忘,纔交你教你記著。\\
        imbe sefu obuci ome ofi, ce teni sefu obuhabi. & 因他可為師,他等纔師之矣。\\
        terei ama jui ah\={u}n deo -i sidende alh\={u}daci ome ofi, irgen teni alh\={u}dambikai. & 因為父子兄弟足法,而后民法之也。
    \end{tabularx}
\end{center}

\subsubsection{“若是、可以、如何”oci, ome, ohode}
\noindent 若是、可以、如何語,oci、ome、ohode。
\begin{center}
    \begin{tabularx}{\textwidth}{XX}
        ere baita be darak\={u} oci ome ohode, uthai galai dara be joo. & 此事若是可以不管,就不必著手。\\
        tere hacin be ak\={u} obuci ome ohode, uthai acara be tuwame icihiyaki. & 那項可免,就請酌量辦理。
    \end{tabularx}
\end{center}

\subsection{“是以、因此”uttu ofi}
\noindent uttu ofi是以因此用。
\begin{center}
    \begin{tabularx}{\textwidth}{XX}
        uttu ofi, irgen -i angga be serem\v{s}erengge, bira be tosoro ci oyonggo, bira ulejefi sendejehede, kokiran ojorongge mujak\={u} ambula. & 是以防民之口,甚於防川,川崩而潰,為傷良多。\\
        uttu ofi narh\={u}\v{s}ame fujurulame baicaci, geren leolen gemu emu adali. & 因此詳加訪查,輿論盡皆相同。
    \end{tabularx}
\end{center}

\subsection{“所以、是故”tuttu ofi}
\noindent tuttu ofi所以、是故多。
\begin{center}
    \begin{tabularx}{\textwidth}{XX}
        tuttu ofi, bira be dasame bahanara urse, sendelefi eyebumbi. irgen be dasame bahanara urse, tucibume gisurebumbi. & 所以善為川者,決之使道路。善為民者,宣之使言。\\
        tuttu ofi ambasa saisa terei ten be baitalarak\={u}ngge ak\={u}. & 是故君子無所不用其極。
    \end{tabularx}
\end{center}

\subsection{“况且”tere anggala}
\noindent tere anggala是况且。
\begin{center}
    \begin{tabularx}{\textwidth}{XX}
        tere anggala, wesihun jalan de bejere yurure jobocun ak\={u}ngge, gemu doro bifi uttu ohongge kai. & 况且盛世無饑寒之累者,皆有道以致之也。\\
        tere anggala, ne juse deote oho niyalma, amaga inenggi geli niyalma de ama ah\={u}n ombi. &况今日之子弟,又為將來之父兄。
    \end{tabularx}
\end{center}

\subsection{“然而”tuttu seme}
\noindent 然而一轉tuttu seme。
\begin{center}
    \begin{tabularx}{\textwidth}{XX}
        niyalma banjire de, emu inenggi seme baitalan ak\={u} obume muterak\={u}, uthai emu inenggi seme ulin ak\={u} oci ojorak\={u}. tuttu seme urunak\={u} ulin be fulukan -i funcebuhe manggi, teni erin ak\={u} -i baitalara de acabuci ombi. &生人不能一日無用,即不可一日無財。然必留有餘之財,而後可供不時之用。\\
        afame bahaci, \v{s}anggaha bi. oihorilafi turibuci weile bi. & 緝捕有賞,疎縱有罰。\\
        h\={u}laha be gidaci, fafun bi bilgan be jurceci, kooli bi. tuttu seme, umesi sain ningge, ga\v{s}an falga banjibure de isirengge ak\={u}. & 諱盜有禁,違限有條,然而最善者,莫如保甲。
    \end{tabularx}
\end{center}

\subsection{“今夫、今有、如今”te bici}
\noindent 今夫今有te bici。
\begin{center}
    \begin{tabularx}{\textwidth}{XX}
        te bici, muji maise be, usefi birembi. & 今夫麰麥, 播種而耰之。\\
        te bici, emu niyalma inenggidari ini adaki booi coko be h\={u}lhambi. & 今有人日攘其鄰之雞者。
    \end{tabularx}
\end{center}

\subsection{“若夫、夫以”te bicibe}
\noindent 若夫、夫以te bicibe。
\begin{center}
    \begin{tabularx}{\textwidth}{XX}
        te bicibe, banin be ak\={u}mbume muterak\={u}ngge, ten -i unenggi -i doro waka kai. & 若夫不能盡其性者,非至誠之道也。\\
        te bicibe abka na -i onco de, aibe baktaburak\={u} jalan fon -i goro de, aibe biburak\={u}. & 夫以天地之大,何所不容。宇宙之遥,何所不有。
    \end{tabularx}
\end{center}

\subsection{“或是”eici, “或有”embici}
\noindent eici或是embici或有。
\begin{center}
    \begin{tabularx}{\textwidth}{XX}
        eici bayan ocibe eici yadah\={u}n ocibe, hesebun forgon wakangge ak\={u}. & 或是富或是貧,莫非命運。\\
        embici sain ningge bisire, embici ehe ningge bisirengge, gemu sukdun salgabun -i haran. & 或有善或有惡,皆緣氣質。
    \end{tabularx}
\end{center}

\subsection{“某、或者”ememu, ememungge}
\noindent 或者ememu與ememungge。
\begin{center}
    \begin{tabularx}{\textwidth}{XX}
        tere niyalma ainahani? ememu fonde getuken, ememu fonde h\={u}lhi. & 那箇人是怎麼了,或者一時明白,或者一時糊塗。\\
        ere giyan naranggi adarame, ememungge inu sembi, ememungge waka sembi. & 這箇理到底怎麼樣,或者說是,或者說不是。
    \end{tabularx}
\end{center}

\subsection{“因為、緣故”jalin, haran, turgun}
\noindent jalin、haran、turgun,因為情由緣故多。
\begin{center}
    \begin{tabularx}{\textwidth}{XX}
        hafan wesime muterak\={u}, ulin madame muterak\={u}ngge, tere hesebun forgno eberi -i jalin kai. & 不能陞官發財,是因為命運不好。\\
        aisilame genggun tucibure jalin, sai\v{s}ame huwekiyebuki seme baime wesimbuhebi. & 因為捐輸,奏請獎勵。\\
        juse deote -i yabun ginggun ak\={u}ngge, an -i ucuri tacihiyarak\={u} -i haran. & 子弟品行不謹,乃平素不教訓的情由。\\
        asihata -i kiceme tacirak\={u}ngge, ai haran bithei amtan be sark\={u} de kai. & 少年不肯勤學,是何情由?不知書中滋味也。\\
        amba baita be mutebume banjinarak\={u}ngge, buya aisi de dosika turgun. & 大事之不得成者,見小利的緣故。\\
        niyalma de enduringge bisire arsari bisirengge, ai turgun seci mergen mentuhun -i banin adali ak\={u} ofi kai. & 人之有聖有凡,是何緣故?乃賢愚之性不同也。
    \end{tabularx}
\end{center}

\subsection{因何/why:ainu, 何必:aiseme, 怎樣做/how:ainara, 莫說:aisere}
\noindent ainu因何aiseme何必,ainara怎樣aisere莫說。
\begin{center}
    \begin{tabularx}{\textwidth}{XX}
        yabufi uttu kejine oho, ainu kemuni isinahak\={u}? & 走了這麼半天了,因何還沒到?\\
        yasa tuwahai uthai isinjiha kai. aiseme emdubei ek\v{s}embini? & 眼看着就到了,何必祇是忙?\\
        ai hacin -i tacibukini, tere \v{s}uwe tacirak\={u} be ainara. & 任憑怎麼教,他索不學怎樣?\\
        arsari niyalma be aiseme, uthai endurin seme inu enteke amba kesi ak\={u} secini. & 莫說是凡人,就是神仙,也沒有這麼大造化呀。
    \end{tabularx}
\end{center}

\subsection{“焉得、何得”aide bahafi, ainambihefi}
\noindent aide bahafi、ainambihefi,焉得、何得、怎麼說。
\begin{center}
    \begin{tabularx}{\textwidth}{XX}
        umai minde alahak\={u} kai. bi aide bahafi sembi? & 並未告訴我,我焉得知道?\\
        jalan -i baita be ambula dulemburak\={u} oci, aide bahafi sembi? & 若不涉獵世務,何得而知?\\
        bi weri gisurehe be donjihak\={u}, ere baita be ainambihefi sara? & 我沒聽見人說,此事怎麼得知?
    \end{tabularx}
\end{center}

\subsection{“由此、由是、於是”ereni, ede, tereci}
\noindent 由此、由是、與於是,於此以此以之多。ereni、ede、tereci,臨文用字自斟酌。
\begin{center}
    \begin{tabularx}{\textwidth}{XX}
        ereni tuwahade, ejen gosingga dasan be yaburak\={u} bime bayambuci, gemu kungdz de ash\={u}burengge. & 由此觀之,君不行仁政而富之,皆棄於孔子者也。\\
        ereni tuwaha de, ambasa saisa ujihengge be seci ombikai. & 由是觀之,君子之所養可知矣。\\
        tacire urse be ereni beyede forgo\v{s}ome baime, ini cisui bahabukini sembi. & 使學者於此返求諸身而自得之。\\
        ereni bada be toktobuci, ai bata gidaburak\={u}, ereni gungge be kiceci ai gungge muteburak\={u}. & 以此制敵,何敵不摧毀,以此圖功,何功不克?\\
        ejen oho niyalma ereni dasan be yabubumbi. amban oho urse ereni irgen be dasambi. & 人君以之出治,人臣以之理民。\\
        tacire urse ede terei tuttu oci acara giyan be baime genehei, goidaha manggi, yongkiyan dursun amba baitalan gemu getukelembi. & 學者於此求其理之當然,久之,全體大用皆明矣。\\
        ede honan -i ba -i ceng halangga juwe fudz tucifi, meng halangga -i ulahangge be bahafi siraha. & 於是河南程氏族兩父子出,而有以接乎孟氏之傳焉。\\
        tereci ebsi geren -i tacin gulu jiramin, dasan tacihiyan sain genggiyen oci, ulhiyen -i umesi taifin -i wesihun de isinahabi. & 由是以來風俗醇厚,政教休明,日臻於郅治之隆矣。\\
        tereci teni deribume salame, tesurak\={u}ngge de niyecetehe. & 於是始興發補不足。
    \end{tabularx}
\end{center}

\subsection{“畢竟、究竟、到底、歸著”naranggi, jiduji}
\noindent naranggi與jiduji, 畢竟、究竟、到底、歸著。
\begin{center}
    \begin{tabularx}{\textwidth}{XX}
        naranggi \v{s}anggaha tusa iletulehe ba ak\={u}. &畢竟未見成效\\
        naranggi oktosilame ebsi ome mutehek\={u}. &到底沒治過來\\
        jiduji umai jortaingge waka.&究竟並非故意\\
        jiduji kemuni terei baili de kai. & 歸着還是虧了他。
    \end{tabularx}
\end{center}


\subsection{“至、極、最、甚、很”umesi, hon, jaci, mujak\={u}, dembei, nokai}
\noindent umesi、hon、jaci、mujak\={u},至極最甚狠過太多。又有dembei、nokai字,量其本文句如何。
\begin{center}
    \begin{tabularx}{\textwidth}{XX}
        umesi oyonggo & 至要緊\\
        umesi buyecuke & 極可愛\\
        umesi yoktak\={u} & 最無味\\
        umesi giyangga & 甚有理\\
        umesi tuwamehangga & 很可觀\\
        hon dabanaha & 過甚矣\\
        hon ojorak\={u} & 太不堪\\
        jaci ubiyada & 甚可惡\\
        jaci dabanahabi. & 太過愈\\
        mujak\={u} mujilen jobombi. & 太勞心\\
        mujak\={u} h\={u}sun baibumbi. & 多費力\\
        mujak\={u} dolo tata\v{s}ambi. & 很掛懷\\
        yabure de dembei mangga & 做之甚難\\
        baita hacin dembei largin & 事務狠繁\\
        tuwara de nokai ja & 看之很易\\
        yan -i ton nokai ujen & 分量最大
    \end{tabularx}
\end{center}

\subsection{“不甚、不很”asuru...V-rak\={u}/ak\={u}}
\noindent 不甚、不很、如何句,asuru下用-rak\={u}、ak\={u}托。
\begin{center}
    \begin{tabularx}{\textwidth}{XX}
        asuru g\={u}nin sith\={u}rak\={u} & 不甚專心\\
        asuru mujin acaburak\={u} & 不很得志\\
        asuru tondo jiramin ak\={u} & 不很忠厚\\
        asuru sure genggiyen ak\={u}. & 不甚聰明
    \end{tabularx}
\end{center}

\subsection{“頻頻、屢屢”emdubei}
\noindent 頻頻、屢屢是emdubei,只是、只管、儘只多。
\begin{center}
    \begin{tabularx}{\textwidth}{XX}
        emdubei hoilalambi. & 頻頻回顧\\
        emdubei kiyakiyambi. & 頻加歎美\\
        emdubei guwelecembi. & 屢屢窺伺\\
        emdubei fujurulambi. & 屢加詢訪\\
        emdubei dabsitambi. & 只是搶辭\\
        emdubei oforodombi. & 只管傳舌\\
        emdubei k\={u}wasadambi. & 儘只誇張
    \end{tabularx}
\end{center}

\subsection{“竟、漠然、全然、漫然”umai, fuhali}
\noindent 並繙umai、fuhali是竟,漠然、全然、漫然說。
\begin{center}
    \begin{tabularx}{\textwidth}{XX}
        umai herserak\={u}. & 並不理會\\
        umai sark\={u}. & 並不知道\\
        fuhali \v{s}elehe. &竟捨了\\
        fuhali waliyaha. & 竟扔子\\
        umai darak\={u}. & 全然不管\\
        umai jobo\v{s}orak\={u} & 漠然無憂\\
        fuhali g\={u}nin de teburak\={u}. & 漫然不關其慮
    \end{tabularx}
\end{center}

\subsection{“盡、都、俱、皆”gemu}
\noindent 盡都俱皆均gemu。
\begin{center}
    \begin{tabularx}{\textwidth}{XX}
        tunggen -i gubci niyalma mujilen -i baita, gemu gisurerak\={u} de baktakabi. & 滿懷心腹事,盡在不言中。\\
        jalan -i eiten baita be, gemu uttu seme tuwaci acambi. & 世間庶務,都做如是觀纔是。\\
        gungge gebu bayan wesihun serengge, gemu tacin fonjin ci baharangge. & 功名富貴,俱從學問中得來。\\
        jalafun aldasi mohon hafu ojorongge, gemu hesebun de bi. & 壽夭窮通,皆在乎命。\\
        tereci g\={u}wa h\={u}sun bume fa\v{s}\v{s}aha geren hafasa be gemu jurgan de afabufi ilgame gisurebu. &其餘出力各員,均着交部議敘。
    \end{tabularx}
\end{center}

\subsection{“全悉、全是、全行”yooni}
\noindent 全悉yooni與wacihiyame。
\begin{center}
    \begin{tabularx}{\textwidth}{XX}
        niyengniyeri bolori juwe forgon -i caliyan jeku be, yooni oncodome guwebuhebi. & 上下兩忙錢糧,全行豁免。\\
        bilume tohorombure, kadalame jafatara babe, tulbime gamarangge yooni nash\={u}n giyan de acanaha. & 撫綏彈壓,調度悉合機宜。\\
        bisirele aniya aniyai baita hacin be, wacihiyame bolgo obume icihiyaha. & 所有歷年案件,全行清理。\\
        inu de adali\v{s}ara waka be wacihiyame geterembuhe. & 悉除夫似是而非。
    \end{tabularx}
\end{center}

\subsection{“一概、普遍”biretei}
\noindent 一概、普徧是biretei,再有gubci與bireme。
\begin{center}
    \begin{tabularx}{\textwidth}{XX}
        biretei ujeleme isebumbi. ainaha seme oncodome guweburak\={u}.& 一概重懲,決不寬貸。\\
        fifaka fosoko ehe h\={u}lha be, biretei gisabume mukiyebuhe. &零星賊匪,概行剿滅。\\
        aga simen \v{s}umin hafufi, gubci bade emu adali yume singgebuhe. & 雨澤深透,普地一律沾濡。\\
        gurun -i gubci fujurungga wen be ambula selgiyehe. & 徧國中雅化覃敷。\\
        dorgici tulergide isitala, bireme gosin jilan be huk\v{s}embi. & 由中達外,普感仁慈。\\
        folofi suwayan hoo\v{s}an de \v{s}uwaselafi, bireme ulhibume selgiye. & 刊刻謄黃,徧行曉諭。
    \end{tabularx}
\end{center}

\subsection{“除...之外”...ci/V-hA/V-rA ci tulgiyen}
\noindent ci tulgiyen是除之外,上遇破字加-ha, -re。
\begin{center}
    \begin{tabularx}{\textwidth}{XX}
        inci tulgiyen, hono geren ci colgorokongge bio ak\={u}n? & 除他之外,還有出眾的沒有?\\
        ereci tulgiyen, jai umai g\={u}wa durun ningge ak\={u} oho. & 除此之外,再沒有別的樣兒的了。\\
        kooli songkoi icihiyaraci tulgiyen, bahaci getukeleme tucibufi hese be baimbi. & 除照例辦理外,相應聲明請旨。\\
        jergi acafi dahabume wesimbureci tulgiyen, neneme harangga bade sakini seme yabubuki sembi. & 除會銜題奏外,先行知照該處可也。
    \end{tabularx}
\end{center}

\subsection{“據稱、據說”V-hAngge bade}
\noindent 由他至此又轉述,-hangge -hengge bade據稱說。
\begin{center}
    \begin{tabularx}{\textwidth}{XX}
        harangga hafan -i ilibuhangge, eme sakdafi ineku baha turgunde, tuhembume ujibureo seme baime alibuhabi. & 據該員稟稱,母老有疾,懇請終養等語。\\
        harangga amban -i wesimbuhengge, esi yargiyan -i bisire arbun dursun be dahame, kesi isibume gisurere be joo. & 據該大臣奏稱,自係實在情形,加恩著毋庸議。\\
        muk\={u}n -i da sai boolaha bade, ere biyade umai nonggiha meitehe baita turgun ak\={u}ngge yargiyan seme boolahabi. & 據族長等報稱,本月並無添裁事故是實。
    \end{tabularx}
\end{center}

\subsection{連屬格助詞de的動詞}
\noindent ala、jabu與fonji、gele、sengguwe、olho、akda、nike、ertu、isina、isibu與jobo,aisila、teisule共-bu-字,上邊一定盡連de。
\begin{center}
    \begin{tabularx}{\textwidth}{XX}
        bi sinde alaha gisun antaka. & 我告訴你的話如何?\\
        si adarame minde jabuha biheni? & 你怎麼答應我來著呢?\\
        i sinde fonjihangge ai baita? & 他問你的是什麼事?\\
        udu fah\={u}n amba seme, takarak\={u} niyalma be acara de gelembi. & 雖然膽大,怕見生人。\\
        baita fa\v{s}\v{s}an be ilibuki seci, joboro suilara de ume sengguwere. & 要立事業,別憚勞苦。\\
        abkai hesebun de olhorongge, tere ambasa saisa dere. & 畏乎天命者,其為君子乎?\\
        horon encehen de akdafi, balai algi\v{s}ame ho\v{s}\v{s}ome batkalambi. & 依著勢力,招搖撞騙。\\
        ambasa saisa an dulimba de nikefi, jalan ci jailafi sabuburak\={u} seme aliyarak\={u}. & 君子依乎中庸,遯世不見知而不悔。\\
        bayan wesihun de ertufi, abkai jaka be doksirame efulembi. & 仗著富貴,暴殄天物。\\
        terei ten de isinafi, abka na de iletulembi. & 及其至也,察乎天地。\\
        beye cihak\={u}ngge be niyalma de ume isibure. & 己所不欲,勿施於人。\\
        tu\v{s}an ak\={u} de ume joboro, adarame ilire de jobo. & 不患無位,患所以立。\\
        jalan de aisilame, irgen de dalambi. & 輔世良民。\\
        guwa bade fe gucu de teisulehengge, inu sebjen wakao? & 他鄉遇故交,不亦樂乎?\\
        erebe minde burak\={u}, hono wede buki sembini? & 這個不給我,還要給誰?
    \end{tabularx}
\end{center}

\subsection{不可接de的方位詞、副詞性質的方位詞,古代方位格-lA、-lAri}
\noindent dele、wala、fejile,dolo、tule與oilo,cala、ebele、amala,-tala、-tele共-tolo。又有-leri, -lori等,字下皆不可接de。除卻dele、ebele下,-i、ni、ci、be都接不得。
\begin{center}
    \begin{tabularx}{\textwidth}{XX}
        nagan -i dele teki. & 請在炕上坐\\
        deretu -i wala faidahangge, ai faidangga? & 桌案下首擺的是什麼擺設?\\
        mooi fejile serguwe\v{s}eci dembei sain, booi dolo bici ebsi gingkambi. & 樹下乘涼很好,在屋子裏怪悶的。\\
        dukai tule tucifi tuwanacina. & 出門外去看看\\
        mukei oilo heni boljon colkon ak\={u}. & 水的浮靣一點波浪沒有。\\
        birai cala ilifi, menerefi ainambi? & 跕在河那邊發愣作什麼?\\
        doohan -i ebele doohanjifi sartabume andubuki. & 過橋這邊來消遣消遣罷。\\
        alin -i amala mudalinafi butha\v{s}anaki sembikai. & 要繞道山後去打獵啊。\\
        daci dubede isitala, amba jurgan halarak\={u}. & 由始至終,大節不渝。\\
        julgeci tetele sain algin buruburak\={u}. & 自古至今,令聞不泯。\\
        utala aniya otolo, dahame yabume heolederak\={u}. & 直至多年,奉行不怠。\\
        morin -i deleri feksibume benebume, ume sartabure. & 馬上飛遞無悞。\\
        yasai juleri karulame acaburangge, calabun ak\={u}. & 眼前報應無差。\\
        mukei oilori neome eyerengge, toktohon ak\={u}. & 水面飄流不定。\\
        mujilen -i dolo hadahai ejefi onggorak\={u}. & 心裏牢記不忘。
    \end{tabularx}
\end{center}

\subsection{emgi, sasa, baru, cihai, funde等需要屬格的後置詞}
\noindent emgi, sasa與baru,還有cihai共funde。遇此等字如何串,上用-i、ni是準則。
\begin{center}
    \begin{tabularx}{\textwidth}{XX}
        jaka -i emgi dekdembi irumbi. & 與物浮沉\\
        ini emgi uhei genembi. & 仝他共往\\
        geren -i sasa getukeleme gisurehe. & 同眾言明\\
        meni sasa boljome jabduha. & 我們大家約定\\
        abkai baru hadahai tuwambi. & 望天上直看\\
        meni baru balai elbenfembi. & 對著我胡說\\
        g\={u}nin -i cihai balai yabumbi. & 任意妄為\\
        ini cihai okini. & 由他去罷\\
        niyalmai funde tulbime bodoho. & 代人籌畫訖\\
        mini funde suilame joboho. & 替我偏勞了
    \end{tabularx}
\end{center}

\subsection{ici, songkoi, jergi, haran, jalin, turgun等接分詞、屬格的後置詞}
\noindent ici、songkoi、jergi,haran、jalin、turgun多。上接-i、ni與-ra、-re,-ha、-he、-ho等也使得。
\begin{center}
    \begin{tabularx}{\textwidth}{XX}
        edun -i ici deli\v{s}embi. & 隨風蕩漾\\
        galai joriha ici. & 順手所指\\
        ere durun -i songkoi weileki. & 照樣作罷\\
        gisurehe songkoi obu. & 著依議\\
        jai jergi hafan -i jergi \v{s}angname nonggi. & 賞加二品銜\\
        deyere furire a\v{s}\v{s}ara ilire jergi jaka. & 飛潛動植之物\\
        yaya baita onggoro manggangga, ejesu ehe -i haran. & 是事愛忘,是記性不好的緣由。\\
        g\={u}nin mujilen toktohon ak\={u}ngge, buyen cihalan bisire haran. & 心神不定,因有嗜好之由。\\
        hesei baicame icihiyabuhangge, ai baitai jalin& 奉旨查辦,所因何事。\\
        bukdari arafi donjibume wesimburengge, turgun be tucibume hese ba baire jalin. & 專摺奏聞,為瀝情請旨事。\\
        buyeme bithe h\={u}larak\={u}ngge, mujilen banuh\={u}n -i turgun. & 不愛念書,乃心裏懶惰的緣由。\\
        hahai erdemu mangga uresh\={u}n ohongge, cohome kiceme urebuhe turgun. & 技藝精熟,乃是勤習之故。
    \end{tabularx}
\end{center}

\subsection{“XX了的話”V-hAde, V-rAde}
\noindent -hade、-hede、-rede等,順著口氣往下說。
\begin{center}
    \begin{tabularx}{\textwidth}{XX}
        gosin be songkolome, jurgan be yabuhade, h\={u}turi ulhiyen -i nonggibumbi. & 履仁由義,則福日增。\\
        gashan be nekuleme, jobolon be sebjelehede, erdemu ulhiyen -i kokirambi. & 幸災樂禍,則德日損。\\
        sain be sai\v{s}ara, ehe be ubiyarade, ambasa saisa ojorak\={u}ngge komso. & 好善惡惡,而不君子者鮮矣。\\ 
        erdemu be fudarara, baili be urgederede, buye niyalma ojorak\={u}ngge ak\={u} kai. & 悖德辜恩,而不小人者未之有也。
    \end{tabularx}
\end{center}

\subsection{V-mbi}
\noindent -mbi 本是有力字。
\begin{center}
    \begin{tabularx}{\textwidth}{XX}
        bithe tacimbi & 習文\\
        cooha urebumbi & 演武\\
        gungge mutebumbi & 功成\\
        gebu ilibumbi & 名立
    \end{tabularx}
\end{center}

\subsection{禁用V-rA的情形}
\noindent 已然推效無-ra、-re。
\begin{center}
    \begin{tabularx}{\textwidth}{XX}
        inenggidari ibefi, biyadari dosifi, tacihai urkuji eldengge genggiyen oho. & 日就月將,學有緝熙於光明。
    \end{tabularx}
\end{center}

\noindent -ra、-re下無-na、-me、-fi。
\begin{center}
    \begin{tabularx}{\textwidth}{XX}
        beyebe tuwanciyara, boo be teksilere de, mujilen be tob g\={u}nin be unenggi obure be da obuhabi. jaka be hafure sarasu de isiburengge, geli terei oyonggongge kai. & 修身齊家,以正心誠意為本,而格物致知,又其要焉者也。
    \end{tabularx}
\end{center}

\subsection{禁用V-hA的情形}
\noindent baha之下忌-ha托。
\begin{center}
    \begin{tabularx}{\textwidth}{XX}
        etuku baha manggi, jai etumbi. buda baha manggi, jai jembi. & 衣服得了再穿,飯食得了再喫。
    \end{tabularx}
\end{center}

\subsection{“總而、總而言之”eicibe}
\noindent 總而神情eicibe,後文末用-ha、-he、-ho。
\begin{center}
    \begin{tabularx}{\textwidth}{XX}
        eicibe yaya kiceme tacire urse, amaga inenggi gemu \v{s}anggaha. & 總而言之凡勤學之士,後來皆成了。\\
        eicibe tulbime bodome bahanarangge, amba muru yooni gungge mutebuhe. & 總而善籌畫者,大抵悉皆成功。\\
        eicibe icihiyanjame gamarangge giyan baha manggi, uthai endebure ba ak\={u} oho. & 總之調停得宜,便無過失矣。
    \end{tabularx}
\end{center}

\subsection{如何adarame, 未必ainahai}
\noindent adarame如何、ainahai未必,總作怎麼、焉、豈、何。近時句尾加ni字,老語-mbi即可托。
\begin{center}
    \begin{tabularx}{\textwidth}{XX}
        adarame icihiyambi. & 如何辦\\
        adarame dasambi. & 怎麼治\\
        ambasa saisa be adarame felehudeci ombi. & 君子焉可誣也\\
        adarame baita ci ukcame mutembi. & 豈能脫然事外哉\\
        adarame mutebure be erembini. & 焉望成乎\\
        ainahai tuttu ak\={u} ni. & 未必不然\\
        ainahai urgunjerak\={u} ome mutembi. & 焉能不喜\\
        ainahai foihorilaci ombini. & 豈可忽也\\
        ainahai gosingga -i teile ni. & 何事於仁
    \end{tabularx}
\end{center}

\subsection{“果然、誠為”mujangga}
\noindent 果然誠為是mujangga,往往句下末位托。
\begin{center}
    \begin{tabularx}{\textwidth}{XX}
        calabun ak\={u} mujangga. & 果然不差\\
        holo ak\={u} mujangga. & 果然不虛\\
        baitangga jaka mujangga. & 誠為梓材\\
        ten -i boobai mujangga. & 誠為至寶
    \end{tabularx}
\end{center}

\subsection{接續連詞jai, geli, sirame}
\noindent 一事方完又一事,jai與geli、sirame。jai字本講再、暨、及、次,又作至於、至若說。復又二字繙geli,繼而旋續是sirame。
\begin{center}
    \begin{tabularx}{\textwidth}{XX}
        harangga ba -i hafan -i oronde, emgeri hafan be sonjofi sindaha. jai erei tucike oron be, hafan tucibufi daiselabuci ojoro ojorak\={u} babe hesei toktobuki. & 該處員缺,已揀員補授矣。再所遺之缺,可否派員署理之處恭候欽定。\\
        ne bisire juwe jergi nonggiha, geli gaifi dahalabure emu jergi nonggiha, uheri nonggiha ilan jergi be, ninggun mudan ejehe jergi obume halaha. & 現有加二級,又有隨帶加一級,共加三級,改為紀錄六次。\\
        baicaci, harangga hafan onggolo duici jergi gemun -i tanggin -i hafan de niyeceme sindara be aliyambihe. sirame jurgan ci ilgame gisurefi, ere hafan sindaha. & 查該員曾以四品京堂候補,續由部議敘,得授令職。\\
        cui dorolon hengkilefi, \emph{\v{s}u ciyang}, jai \emph{be ioi} de anah\={u}njaha. & 垂拜稽首,讓于殳斨暨伯與。\\
        ere dorolon, goloi beise daifusa jai hafasai geren niyalma de isinahabi. &斯禮也,達乎諸侯大夫及士庶人。\\
        ineku aniya jurgan de afabufi kimcime gisurebufi, jai aniya deribume yabubuki sembi. & 是年交部核議,次年舉行可也。\\
        jai suweni geren cooha irgen, tacik\={u}i yamun -i ujen be sark\={u} oci, suwende daljak\={u} seme g\={u}nihabi. & 至於爾兵民,不知學校為重,以為與爾等無異。\\
        jai ama de da jui bici, booi dalahangge seme tukiyembi. deo de ah\={u}ngga ah\={u}n bici, booi ah\={u}cilahangge seme wesihulembi. &至若父有家子,稱曰家督,弟有伯兄,尊曰家長。\\
        geli \v{s}i halangga -i arahangge be gaifi, terei largin facuh\={u}n be meitehe. & 復取石氏書,刪其繁亂。\\
        tuktan de angga aljaha bihe, sirame gisun aifuha. & 始而應允,繼而食言。\\
        dade fideme kadalara amban sindaha, sirame jiyanggiy\={u}n -i doron -i baita be daiselaha. & 初授提督,旋署將軍印務。
    \end{tabularx}
\end{center}

\subsection{“還是、仍、尚、猶、嘗”kemuni}
\noindent 還是常用kemuni,仍尚猶嘗併字多。
\begin{center}
    \begin{tabularx}{\textwidth}{XX}
        kemuni adarame gamaci sain ni? & 還是怎麼之好呢?\\
        kemuni uttu icihiyambihe? & 常常這麼辦來著?\\
        kemuni ak\={u} ume muterak\={u}. & 仍不能免\\
        kemuni tolgin de bi. & 尚在夢中\\
        kemuni hafukiyame ulhihek\={u}. & 猶未徹悟\\
        kemuni dengjan dabume sara\v{s}ambihe. & 嘗作秉燭遊\\
        sunjaci jergi jingse \v{s}angname hadabureci tulgiyen, kemuni tojin funggala \v{s}angname halame hadabu. & 賞給五品頂戴,並賞換花翎。\\
        kemuni g\={u}nici, doroi tacin -i sekiyen, ten -i enduringge ci ulahabi. & 嘗思道學之源,傳於至聖。
    \end{tabularx}
\end{center}

\subsection{“方、將、剛、纔、始、乃、然後、而後”teni}
\noindent teni方將剛纔始乃,又作然後、而後說。
\begin{center}
    \begin{tabularx}{\textwidth}{XX}
        teni mejige baha. & 方得信\\
        teni amasi mariha. & 將回來\\
        teni boode isinjiha. & 剛到家\\
        teni oksome bahanambi. & 纔會走\\
        teni sith\={u}me deribuhe. & 始發憤\\
        julgei tacihiyan be taciha manggi, teni bahame mutembi. & 學於古訓,乃有獲。\\
        booi niyalma de acabungga ohode, teni gurun -i niyalma be tacibuci ombi. & 宜其家人,而后可以教國人。\\
        taciha manggi teni tesurak\={u} be sambi. & 學然後知不足
    \end{tabularx}
\end{center}

\subsection{“祇顧、奈、惟”damu}
\noindent damu作祇顧奈惟第,但只無如字意多。
\begin{center}
    \begin{tabularx}{\textwidth}{XX}
        damu gubci mederi -i dorgi tulergide, amba wen bireme selgiyekini sembi. & 祇期薄海內外,大化翔洽。\\
        damu ga\v{s}an falgai dolo banjiha fusekengge ulhiyen -i geren ofi, boo g\={u}wa fiheme adambi. & 顧鄉黨中生齒日繁,比閭相接。\\
        damu jalan -i niyalma liyeliyefi ulhirak\={u}, yala nasacuka kai. & 奈世人迷而不悟,良可悲夫。\\
        damu yekengge niyalma ejen -i mujilen -i waka be tuwancihiyame mutere be oyonggo obuhabi. & 惟大人惟能格君心之非。\\
        damu dahame yaburengge inenggi goidaha manggi, untuhun gisun obufi tuwarak\={u} sembi. & 第恐遵行日久,視為具文。\\
        damu jug\={u}n de yaburengge, gemu h\={u}turi cinggiya -i niyalma kai. & 但是登途者,都是福薄人。\\
        damu untuhun gebu be baire gojime, yargiyan tusa be bairak\={u}. & 只邀虛名,不求實效。\\
        damu inenggi goidafi eimeme deribufi, jug\={u}n -i andala aldasilaha. & 無如日久生厭,半途而廢。
    \end{tabularx}
\end{center}

\subsection{“而已”wajiha, “而已矣”wajihabi}
\noindent wajiha作而已用,wajihabi而已矣說。wajiha上連ci字,wajihabi上接de。
\begin{center}
    \begin{tabularx}{\textwidth}{XX}
        manggai baita be wacihiyaci wajiha, aiseme yongkiyara be baimbini? & 無非了事而已,何必求全。\\
        manggai uttu gamaci wajiha, hono adarame icihiyaki sembini. & 不過如此而已,還要怎麼辦呢?\\
        fudz -i doro, tondo giljan de wajihabi. & 夫子之道忠恕而已矣。\\
        tacin fonjin -i doro, g\={u}wa de ak\={u}, tere turibuhe mujilen be baire de wajihabi. & 學問之道,無他,求其放心而已矣。
    \end{tabularx}
\end{center}

\section{結語}
\noindent 今將清文之法則,集成數語利初學。\\
\noindent bi temgetu jorin -i bithe banjibuha amala, geli untuhun hergen -i uculen bihe be dahame, gisun holboro ici ak\={u} oci ojorak\={u} seme g\={u}nime, tereci ainame emu udu gisun banjibufi, tuktan tacire urse de ulhibuki.\\
\noindent 余於指南編竟,因思虛字既已有詞,過文豈可無具,於是杜撰數語,以示初學。
\begin{center}
    \begin{tabularx}{\textwidth}{XX}
        -i字之上接整字,&無論單連都使得。\\
        ni接整字-mbi等bi,&-ka -ke -ko與-ha -he -ho。\\
        de be ci字三個上,&除去-i ni與-fi -me。\\
        更有破字-la -le -lo,&-lori -leri -hai -hei字多。\\
        -mbi -habi -hebi等字外,&其餘整破都接得。\\
        kai字上除-i 共-me,&-tala -tele 共-tolo。\\
        還有-ra -re -ro字外,&整破不拘皆連得。\\
        所有長音各-k\={u}字,&亦如整字用法活。\\
        sembi等字承何語,&整字末尾結文多。\\
        此是清文總規矩,&寄語學人自參酌。\\
    \end{tabularx}
\end{center}
\noindent 歌曰:
\begin{center}
    \begin{tabularx}{\textwidth}{XX}
        清文一道卻如何,&整上破下是文波。\\
        de be ci -i 中間串,&ni kai nio等末尾多。\\
        實義虛文須仔細,&已然未然要斟酌。\\
        平鋪變繙與調換,減多增少細琢磨。
    \end{tabularx}
\end{center}
\noindent ere emu meyen ucun de, neneme manju bithei ici be gisurehebi. sirame manjurame ubaliyambure fakjin be gisurehebi. tacire urse erei songkoi tacime ohode, ulhiyen -i inenggidari ibenefi biyadari dosinafi, micihiyan ci \v{s}umin de isinaci ombi. \\
\noindent 此一歌,先言清文之則,次言繙譯之法,學者依此學去,庶幾日就月將,可由淺而造其深也。
\end{document}